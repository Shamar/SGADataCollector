\chapter{Discussione}

La terapia con ormone della crescita ricombinante nei bambini nati SGA si pone come obiettivo la prevenzione del deficit staturale in età adulta.
Lo scopo della mia tesi è verificare se tale obiettivo è stato raggiunto su ventotto soggetti SGA seguiti presso la Struttura Semplice di Auxologia del Dipartimento di Discipline Pediatriche e dell'Adolescenza dell'Università  di Torino.

La maggioranza  degli studi finora condotti con questo scopo è multicentrica ed ha incluso i pazienti con la  sindrome di Silver Russell. Nella mia tesi ho escluso questi soggetti perché aventi caratteristiche cliniche e auxologiche peculiari che li differenziano dai bambini SGA non sindromici. Tutti i pazienti della casistica in esame sono stati seguiti nel tempo dagli stessi operatori esperti e ciò permette di disporre di misure accurate e confrontabili fra loro. Va sottolineato, inoltre, che la casistica di questo studio è interamente italiana, a differenza degli altri studi, che sono stati svolti per lo più su bambini dell'Europa settentrionale, i quali hanno un potenziale genetico di crescita differente.

I soggetti da me esaminati hanno raggiunto una statura definitiva media pari a -2 SDS, ponendosi al limite inferiore della norma per la popolazione generale. Il risultato, apparentemente poco incoraggiante, è da ritenersi soddisfacente se si considerano la statura  media all'inizio della terapia (-2,9 SDS) nettamente deficitaria, ed il basso potenziale genetico (in media il centro della fascia-bersaglio parentale, MPH, era pari a -1,6 SDS). I pazienti hanno infatti guadagnato circa una deviazione standard ed hanno raggiunto l'MPH. I risultati ottenuti si riferiscono ai valori medi, che ovviamente includono i bambini cresciuti molto e quelli cresciuti poco. 

Dall'analisi statistica emerge che la risposta alla terapia sembra essere influenzata da alcune variabili.
Una di queste è l'età di inizio del trattamento, in media dieci anni (inversamente correlata al guadagno staturale, $p < 0,001$). Si tratta di un'età piuttosto avanzata, anche se simile alle altre casistiche\cite{coutant1998short} \cite{zucchini2001final}. Come già esposto nell'introduzione, l'infanzia è il periodo più importante nella crescita di un individuo. Pertanto un inizio del trattamento tardivo, di poco precedente l'ingresso in pubertà o a sviluppo già cominciato, comporta un minor guadagno staturale. 

La statura ad inizio trattamento è risultata inversamente correlata al guadagno staturale (con $p < 0,001$) : i bambini più bassi sembrano rispondere meglio alla terapia. Questa correlazione può almeno in parte essere giustificata dalla formula che si utilizza per calcolare il guadagno staturale (esso corrisponde alla differenza fra la statura finale e la statura all'inizio della terapia). Altri studi hanno riportato la medesima correlazione\cite{de2005growth} \cite{de2000growth}.

Un'altra variabile altamente predittiva è l'altezza all'inizio della pubertà (direttamente correlata al guadagno, $p < 0,001$). Ciò conferma quanto osservato da Prader\cite{gasser1985human}, Stanhope\cite{stanhope1988new} e Largo \cite{gasser2001growth} e riportato nell'introduzione: 
 all'inizio della pubertà le ragazze hanno già realizzato circa l' 85\% della statura definitiva, mentre i ragazzi il 90\%. Pertanto, come emerge dal'analisi statistica, un'alta statura all'inizio dello sviluppo porta ad un'alta statura finale. Infatti la paziente B.M., che ha tratto il maggior guadagno in termini di SDS (3,5), ha iniziato la pubertà ad una statura di 141,5 cm (limite superiore dell'intervallo entro cui varia la statura all'inizio dello sviluppo per le femmine della nostra casistica).
 
Ne consegue che ai fini del guadagno staturale sono cruciali gli anni prepuberali di terapia con rGH. Infatti Dahlgren et al. hanno riscontrato incrementi di 1,7 SDS sulla statura adulta con terapia iniziata almeno due anni prima della pubertà\cite{dahlgren2005final}. Tuttavia l'analisi statistica non ha confermato una diretta correlazione tra la durata del trattamento prepuberale e il guadagno staturale. Verosimilmente, però, si tratta di un risultato fortemente condizionato dal fatto che la grande maggioranza dei nostri pazienti è stata trattata per un periodo di tempo estremamente breve prima dello sviluppo puberale. In particolare cinque di loro hanno intrapreso la terapia a pubertà già iniziata.

Per quanto riguarda l'entità di scatto puberale abbiamo riscontrato che la terapia proseguita per tutta la durata della pubertà è stata efficace: lo scatto è stato di 27 cm nei maschi e 21 cm nelle femmine, incrementi elevati se confrontati con la popolazione generale\cite{tanner1990foetus}. 

Inoltre il trattamento con rGH non ha indotto l'anticipo nella comparsa dei caratteri sessuali secondari, anzi l'età media di inizio per i maschi è stata di 12,4 anni e per le femmine di 11,3 anni, ossia un anno più tardi rispetto alla popolazione generale italiana\cite{benso1989distribution}.

Non sono emerse differenze fra pazienti con insufficienti e normali livelli di ormone somatotropo.

\chapter{Considerazioni conclusive}

La mia tesi consente di ipotizzare che il trattamento con rGH abbia permesso ai bambini SGA iposomatici di raggiungere una statura normale o comunque compatibile con il potenziale genetico, obiettivo che avrebbe potuto non essere realizzato in assenza dell'intervento terapeutico. 

Infatti la bassa statura adulta è la più frequente complicanza di iposviluppo fetale: si stima che un quarto degli adulti di bassa statura sia nato SGA. 

I pazienti SGA costituiscono un gruppo eterogeneo per quanto riguarda le cause di scarsa crescita e la risposta alla terapia è estremamente variabile:  alcuni di loro cresceranno molto e altri (sindromici, affetti da displasie scheletriche sfumate), definiti \textit{poor responder}\cite{bang2011comparison}, presenteranno un guadagno staturale più modesto.

Nella mia tesi il guadagno staturale è risultato significativamente correlato con la statura di inizio terapia e all'inizio dello sviluppo puberale, con l'età di inizio terapia e con la durata prepuberale di trattamento.

Sono comunque necessari ulteriori sforzi per identificare elementi predittivi di efficacia della terapia.

Le recenti disposizioni contenute nella nota AIFA 39 (in Appendice), che anticipano il trattamento con rGH  a quattro anni di vita, potrebbero aiutare a prevenire il deficit staturale dell'adulto.
