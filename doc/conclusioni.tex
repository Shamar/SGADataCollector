\chapter{Discussione e conclusioni}

% tutti gli studi finora condotti comprendevano anche i Silver Russell
% FH, guadagno in SDS, SDS dal MPH
% deficit, non deficit, età inizio terapia,TH ( dire quali di queste sono variabili  predittive secondo KIGS e Hokken Koelega)
% proposta di follow-up(vedere presentazione su SGA e sottolineare importanza inizio precoce perchè 2/3 SGA sono poor responders, come indicato su Horm  Res pag 100

Tutti gli studi finora condotti sui bambini SGA in terapia con ormone della crescita e giunti alla statura finale hanno incluso i pazienti con sindrome di Silver Russell e sono multicentrici. Il presente studio ha escluso i pazienti con sindrome di Silver Russell perchè aventi caratteristiche cliniche e auxologiche peculiari che li differenziano dai soggetti SGA non sindromici. Inoltre tutti i pazienti della casistica in esame sono stati seguiti dallo stesso Centro di Auxologia dell'Istituto di Discipline Pediatriche dell'Università degli Studi di Torino. Ciò ha permesso di didsporre di misure accurate e confrontabili fra loro, perchè eseguite dagli stessi operatori qualificati. Va sottolineato che la casistica di questo studio è interamente piemontese, a differenza degli altri studi, che sono stati svolti per lo più su bambini dell'Europa settentrionale, i quali hanno un potenziale genetico di crescita differente.

I pazienti del presente studio hanno raggiunto una statura definitiva media pari a -2 SDS. Il risultato, apparentemente poco incoraggiante, è buono se si considerano l'altezza ad inizio terapia (in media -2,8 SDS) ed il potenziale genetico ( in media il centro della fascia-bersaglio parentale, MPH, era pari a -1,6 SDS). Infatti i pazienti hanno guadagnato mediamente circa una deviazione standard ed in media hanno pressochè raggiunto l'MPH. 

Non sono emerse differenze tra i pazienti affetti da deficit di GH e quelli con normali livelli di ormone somatotropo in termini di guadagno staturale e distanza dal centro della facia-bersaglio parentale. Indubbiamente la statura dei genitori ha avuto un peso non indifferente nel determinare la statura finale. Ad esempio i genitori di E.A. erano molto bassi (il padre era alto 144,5 cm, mentre la madre presentava una statura inferiore a 140 cm) ed E.A., pur traendo beneficio dalla terapia con rGH (ha guadagnato 0,6 SDS ed ha pressochè raggiunto l'MPH), non ha normalizzato la sua statura rispetto alla popolazione generale.
Il centro del bersaglio parentale è decritto come variabile predittiva della statura definitiva dei bambini SGA in terapia con rGH anche in un recente studio condotto da Hokken-Koelega\cite{de2008prediction}.

Un'importante variabile che ha influenzato in misura non irrilevante l'esito della terapia è stata l'età di inizio del trattamento. Infatti i media i pazienti hanno intrapreso la terapia ad un'età di dieci anni,così beneficiando di un breve periodo di trattamento prepuberale (in media 1,9). L'infanzia è il periodo più importante per la crescita di un individuo: come esposto nell'introduzione, la massima velocità di crescita è raggiunta nei primi anni di vita e all'inizio della pubertà le ragazze hanno già realizzato circa il 85\% della statura definitiva, mentre i ragazzi il 90\%. Infatti la paziente B.M.,che ha tratto il maggior guadagno in termini di SDS (3,5), ha effettuato circa sette anni di trattamento prima dello sviluppo puberale. Queste osservazioni avvalorano l'attuale nota 39 AIFA (vedi Appendice), secondo cui la terapia   