\chapter{Discussione e conclusioni}

La terapia con ormone della crescita ricombinante nei bambini nati SGA si pone come obiettivo la prevenzione del deficit staturale in età adulta.
Lo scopo della mia tesi è verificare se tale obiettivo è raggiungibile.

Tutti gli studi finora condotti sui bambini SGA in terapia con ormone della crescita e giunti alla statura finale hanno incluso i pazienti con sindrome di Silver Russell e sono multicentrici. Il presente studio ha escluso i pazienti con sindrome di Silver Russell perché aventi caratteristiche cliniche e auxologiche peculiari che li differenziano dai soggetti SGA non sindromici. Inoltre tutti i pazienti della casistica in esame sono stati seguiti dalla stessa Struttura Semplice di Auxologia del Dipartimento di Discipline Pediatriche e dell'Adolescenza dell'Università  di Torino. Ciò ha permesso di disporre di misure accurate e confrontabili fra loro, perché eseguite dagli stessi operatori. Va sottolineato che la casistica di questo studio è interamente piemontese, a differenza degli altri studi, che sono stati svolti per lo più su bambini dell'Europa settentrionale, i quali hanno un potenziale genetico di crescita differente.

I soggetti da me esaminati hanno raggiunto una statura definitiva media pari a -2 SDS, cioè al limite inferiore della norma per la popolazione generale. Il risultato, apparentemente poco incoraggiante, è buono se si considerano la statura all'inizio della terapia (in media -2,8 SDS)nettamente deficitaria ed il potenziale genetico (in media il centro della fascia-bersaglio parentale, MPH, era pari a -1,6 SDS). Infatti i pazienti hanno guadagnato mediamente circa una deviazione standard ed hanno raggiunto l'MPH. I risultati ottenuti si riferiscono ai valori medi, che ovviamente includono i bambini cresciuti molto e quelli cresciuti poco. 

Dall'analisi statistica emerge che la risposta alla terapia sembra essere influenzata da alcune variabili.
Una delle variabili che hanno influenzato in misura significativa l'esito della terapia è stata l'età di inizio del trattamento, in media dieci anni (inversamente correlata al guadagno staturale, $p < 0,001$). Si tratta di un'età piuttosto avanzata anche se simile alle altre casistiche\cite{coutant1998short} \cite{zucchini2001final}. Come già esposto nell'introduzione, l'infanzia è il periodo più importante nella crescita di un individuo. Pertanto un inizio del trattamento tardivo, di poco precedente l'ingresso in pubertà o a sviluppo già cominciato, comporta un minor guadagno staturale. A conferma di ciò uno studio di recente pubblicazione condotto da Toumba et al. ha riportato un significativo miglioramento della statura definitiva di un gruppo di bambini affetti dalla sindrome di Silver Russell (i quali costituiscono una sottopolazione di SGA caratterizzata da una pessima prognosi staturale) che avevano intrapreso la terapia con rGH ad un'età media di cinque anni.%citare Toumba 

L'altezza ad inizio trattamento è risultata inversamente correlata al guadagno staturale: i bambini più bassi sembrano rispondere meglio alla terapia. Questa correlazione può almeno in parte essere giustificata dalla formula che si utilizza per calcolare il guadagno staturale (esso corrisponde alla differenza fra la statura finale e la statura all'inizio della terapia). Altri studi hanno riportato la medesima correlazione\cite{de2005growth}%citare anche altro studio di de Zegher (citazione biblio 15 RSS)

Un'altra variabile altamente predittiva è l'altezza all'inizio della pubertà (direttamente correlata al guadagno, $p < 0,001$). Ciò conferma quanto osservato da Prader, Stanhope e Largo  e riportato nell'introduzione: 
 all'inizio della pubertà le ragazze hanno già realizzato circa l' 85\% della statura definitiva, mentre i ragazzi il 90\%. Pertanto, come emerge dal'analisi statistica, un'alta statura all'inizio dello sviluppo porta ad un'alta statura finale. Infatti la paziente B.M., che ha tratto il maggior guadagno in termini di SDS (3,5), ha iniziato la pubertà ad una statura di 141,5 cm (limite superiore dell'intervallo entro cui varia la statura all'inizio dello sviluppo puberale per le femmine della nostra casistica). 
%Queste osservazioni avvalorano l'attuale nota 39 AIFA (vedi Appendice), secondo cui la terapia con rGH può essere intrapresa dai bambini SGA che non hanno normalizzato la statura all'età di quattro anni.

Ne consegue che ai fini del guadagno staturale sarebbero cruciali gli anni prepuberali di terapia con rGH. Infatti Dahlgren et al. hanno riscontrato incrementi di 1,7 SDS sulla statura adulta con terapia iniziata almeno due anni prima della pubertà\cite{dahlgren2005final}. Tuttavia l'analisi statistica non ha confermato una diretta correlazione tra la durata del trattamento prepuberale e il guadagno staturale. Verosimilmente, però, si tratta di un risultato fortemente condizionato dal fatto che la grande maggioranza dei nostri pazienti è stata trattata per un periodo di tempo estremamente breve prima dello sviluppo puberale. In particolare cinque dei nostri pazienti hanno intrapreso la terapia a pubertà già iniziata.

Non sono emerse differenze fra coloro che erano deficitari di GH  e coloro che presentavano normali livelli di ormone della crescita.

Il trattamento con rGH non ha indotto l'anticipo nella comparsa dei caratteri sessuali secondari, anzi l'età media di inizio per i maschi è stata di 12,4 anni e per le femmine di 11,3 anni, cioè un anno più tardi rispetto alla popolazione generale italiana.% citare benso e Lorenzini

La terapia proseguita per tutta la durata della pubertà è stata efficace in termini di statura guadagnata con lo scatto puberale: 27 cm nei maschi e 21 cm nelle femmine. Per confronto, nella poplazione generale lo scatto di crescita puberale medio è pari a 23 cm per i maschi e 20 cm per le femmine.  \cite{tanner1990foetus} 


  