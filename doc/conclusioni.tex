\chapter{Discussione e conclusioni}

% tutti gli studi finora condotti comprendevano anche i Silver Russell
% FH, guadagno in SDS, SDS dal MPH
% deficit, non deficit, età inizio terapia,TH ( dire quali di queste sono variabili  predittive secondo KIGS e Hokken Koelega)
% proposta di follow-up(vedere presentazione su SGA e sottolineare importanza inizio precoce perché 2/3 SGA sono poor responders, come indicato su Horm  Res pag 100

Tutti gli studi finora condotti sui bambini SGA in terapia con ormone della crescita e giunti alla statura finale hanno incluso i pazienti con sindrome di Silver Russell e sono multicentrici. Il presente studio ha escluso i pazienti con sindrome di Silver Russell perché aventi caratteristiche cliniche e auxologiche peculiari che li differenziano dai soggetti SGA non sindromici. Inoltre tutti i pazienti della casistica in esame sono stati seguiti dallo stesso Centro di Auxologia dell'Istituto di Discipline Pediatriche dell'Università degli Studi di Torino. Ciò ha permesso di disporre di misure accurate e confrontabili fra loro, perché eseguite dagli stessi operatori. Va sottolineato che la casistica di questo studio è interamente piemontese, a differenza degli altri studi, che sono stati svolti per lo più su bambini dell'Europa settentrionale, i quali hanno un potenziale genetico di crescita differente.

I pazienti del presente studio hanno raggiunto una statura definitiva media pari a -2 SDS, cioè al limite inferiore della norma per la popolazione generale. Il risultato, apparentemente poco incoraggiante, è buono se si considerano la statura all'inizio della terapia (in media -2,8 SDS)nettamente deficitaria ed il potenziale genetico (in media il centro della fascia-bersaglio parentale, MPH, era pari a -1,6 SDS). Infatti i pazienti hanno guadagnato mediamente circa una deviazione standard ed hanno raggiunto l'MPH. I risultati ottenuti si riferiscono ai valori medi, che ovviamente includono i bambini cresciuti molto e quelli cresciuti poco. 

Dall'analisi statistica emerge che la risposta alla terapia sembra essere influenzata da alcune variabili.
Una delle variabili che hanno influenzato in misura significativa l'esito della terapia è stata l'età di inizio del trattamento, in media dieci anni (inversamente correlata al guadagno staturale, $p < 0,001$). Si tratta di un'età piuttosto avanzata anche se simile alle altre casistiche\cite{coutant1998short} \cite{zucchini2001final}. La crescita di recupero nei soggetti SGA avviene nei primi due anni di vita e se assente o incompleta comporta bassa statura adulta. Come già esposto l'infanzia è il periodo più importante nella crescita di un individuo. 

Un'altra variabile altamente predittiva è l'altezza all'inizio della pubertà (direttamente correlata al guadagno, $p < 0,001$). Ciò conferma quanto osservato da Prader, Stanhope e Largo: 
 all'inizio della pubertà le ragazze hanno già realizzato circa l' 85\% della statura definitiva, mentre i ragazzi il 90\%. Pertanto, come emerge dal'analisi statistica, un'alta statura all'inizio dello sviluppo porta ad un'alta statura finale. Infatti la paziente B.M., che ha tratto il maggior guadagno in termini di SDS (3,5), ha iniziato la pubertà ad una statura di 141,5 cm (limite superiore dell'intervallo entro cui varia . 
Queste osservazioni avvalorano l'attuale nota 39 AIFA (vedi Appendice), secondo cui la terapia con rGH può essere intrapresa dai bambini SGA che non hanno normalizzato la statura all'età di quattro anni.


  