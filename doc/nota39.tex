\chapter[Nota 39 AIFA - 18 Novembre 2010]
	{Nota 39 AIFA\\[.5ex]
	\normalsize\textit{18 Novembre 2010, GAZZETTA UFFICIALE DELLA REPUBBLICA ITALIANA Serie generale - n. 270}}

\section*{Ormone della crescita (somatotropina), Allegato 1}

La prescrizione a carico del SSN, su diagnosi e piano terapeutico di centri specializzati, Universit\`a, Aziende
Ospedaliere, Aziende Sanitarie, IRCCS, individuati dalle Regioni e dalle Province autonome di Trento e
Bolzano, \`e limitata alle seguenti condizioni:

\subsection*{Et\`a neonatale}

in individui con evidenza neuroradiologica di malformazioni/lesioni ipotalamo ipofisarie e segni 
clinico laboratoristici compatibili con la diagnosi di panipopituitarismo congenito. Tale trattamento 
dovrebbe essere proseguito ininterrottamente almeno per i primi due anni di vita. Successivamente, 
previa interruzione della terapia con GH di durata non superiore a tre mesi, dovrebbe essere eseguita 
una rivalutazione del profilo auxologico, ormonale e laboratoristico finalizzata a determinare 
l’opportunità e la modalità della prosecuzione del trattamento GH

\subsection*{Et\`a evolutiva}
\begin{itemize}
\item 	bassa statura da deficit di GH definito dai seguenti parametri clinico - auxologici e di laboratorio:
	\begin{enumerate}
	\renewcommand{\theenumi}{\Roman{enumi}:}
	\renewcommand{\labelenumi}{\theenumi}
	\renewcommand{\theenumii}{\alph{enumii})}
	\renewcommand{\labelenumii}{\theenumii}
	\item Parametri clinico - auxologici:
		\begin{enumerate}
			\item 	statura $<$ -3DS oppure statura $<$ -2DS e velocit\`a di crescita/anno $<$ -1DS rispetto alla norma per et\`a e
				sesso, misurata a distanza di almeno 6 mesi con le stesse modalit\`a;
		\end{enumerate}
		\textit{oppure}
		\begin{enumerate}
			\setcounter{enumii}{1}
			\item 	velocit\`a di crescita/anno $<$ -2DS o $<$ -1,5 DS dopo 2 anni consecutivi, anche in assenza di bassa statura;
				nei primi 2 anni di vita, sar\`a sufficiente fare riferimento alla progressiva decelerazione della velocit\`a di
				crescita (la letteratura non fornisce a riguardo dati definitivi in termini di DS);
		\end{enumerate}
		\textit{oppure}
		\begin{enumerate}
			\setcounter{enumii}{2}
			\item 	malformazioni/lesioni ipotalamo-ipofisario dimostrate a livello neuroradiologico o difetti ipofisari
				multipli che comportino deficit di GH accertato in base ad una delle modalit\`a del punto II
		\end{enumerate}
	\item Parametri di laboratorio:
		\begin{enumerate}
			\setcounter{enumii}{0}
			\item 	risposta di GH $<$ 10 \unit{\micro g}/L a due test farmacologici eseguiti in giorni differenti (la risposta ad un solo test
				farmacologico $>$ 10 \unit{\micro g}/L esclude la diagnosi di deficit di GH) 	
		\end{enumerate}
		\textit{oppure}
		\begin{enumerate}
			\setcounter{enumii}{1}
			\item 	risposta di GH $<$ 20 \unit{\micro g}/L nel caso uno dei due test impiegati sia GHRH + arginina o GHRH
				+piridostigmina
		\end{enumerate}
	\end{enumerate}
\end{itemize}

\subsection*{Altre condizioni dove il trattamento con rGH viene concesso in et\`a pediatrica:}

\begin{itemize}
\item sindrome di Turner citogeneticamente dimostrata;
\item deficit staturale nell'insufficienza renale cronica;
\item soggetti prepuberi affetti dalla sindrome di Prader Willi (PWS), geneticamente dimostrata, con Indice di Massa Corporea o Body Mass
Index (BMI) $<$ 95°, normale funzionalit\`a respiratoria, non affetti da sindrome dell'apnea ostruttiva nel
sonno.
\item Bambini nati piccoli per l'et\`a gestazionale (SGA - Small for Gestational Age) con et\`a uguale o superiore a
4 anni.
\end{itemize}

Per poter accedere al trattamento con GH in individui nati SGA \`e necessario rispondere ai seguenti criteri:
\begin{itemize}
\item peso alla nascita nei nati singoli uguale o inferiore a –2 DS ($<$ 3° centile) per l'et\`a gestazionale, basato
sulle tabelle di Gagliardi (L. Gagliardi et Al. “Standard antropometrici neonatali prodotti dalla task-
force della Societ\`a Italiana di Neonatologia e basati su una popolazione italiana nord-orientale” Riv.
Ital. Pediatr. (IJP) 1999; 25: 159-169) e comunque inferiore a 2500 gr.
\item et\`a al momento della proposta di somministrazione del GH uguale o superiore ai 4 anni;
\item statura inferiore o uguale a –2.5 DS e velocit\`a di crescita inferiore al 50° centile.
\item Autorizzazione delle Commissione Regionale preposte alla sorveglianza epidemiologica ed al
monitoraggio dell'appropriatezza del trattamento con GH
\end{itemize}

Considerando la relativa limitata esperienza del trattamento con GH negli SGA in Italia, l'autorizzazione al
trattamento con rGH in soggetti SGA \`e concessa per 2 anni previa verifica ed autorizzazione da parte delle
Commissioni Regionali preposte alla sorveglianza epidemiologica ed al monitoraggio dell'appropriatezza del
trattamento con GH appartenenti alla residenza del paziente. Dopo 2 anni di terapia, il proseguimento
terapeutico potr\`a essere nuovamente autorizzato dalle Commissioni Regionali dopo una verifica dei risultati
clinici ottenuti nel periodo di trattamento.

In caso di mancata istituzione della commissione regionale, la proposta al trattamento con GH da parte del
centro prescrittore dovr\`a essere indirizzata alla Commissione preposta alla sorveglianza epidemiologica ed al
monitoraggio dell'appropriatezza del trattamento con GH presso l'Istituto Superiore di Sanit\`a, che dovr\`a dare
una risposta al centro prescrittore entro giorni trenta dal ricevimento della richiesta.

\subsection*{Et\`a di transizione}

Viene definita et\`a di transizione quella compresa tra il momento del raggiungimento della statura definitiva del
soggetto trattato e l'et\`a di 25 anni.
Al raggiungimento della statura definitiva non \`e più indicata la terapia con GH nelle seguenti patologie:
\begin{itemize}
\item sindrome di Turner;
\item insufficienza renale cronica
\item sindrome di Prader Willi;
\item Soggetti nati piccoli per et\`a gestazionale (SGA).
\end{itemize}

Al raggiungimento della statura definitiva la terapia con GH può essere proseguita senza ulteriori rivalutazioni
nelle seguenti patologie:
\begin{itemize}
\item deficit di GH causato da mutazione genetica documentata
\item panipopituitarismo con difetto di secrezione ormonale multiplo di almeno tre ormoni ipofisari.
\end{itemize}

Alraggiungimento della statura definitiva la terapia con rGH negli altri soggetti con deficit di GH può essere
proseguita solo se presentano dopo almeno un mese dalla sospensione del trattamento sostitutivo con rGH:
\begin{itemize}
\item risposta di GH $<$6 \unit{\micro g}/L dopo ipoglicemia insulinica (ITT);
\end{itemize}
oppure
\begin{itemize}
\item risposta di GH $<$ 19 \unit{\micro g}/L dopo test farmacologico con GHRH + Arginina.
\end{itemize}

\subsection*{Et\`a adulta}

\`E indicata la terapia con rGH in soggetti adulti, di et\`a superiore a 25 anni, con livelli di GH allo stimolo con
ipoglicemia insulinica $<$3 \unit{\micro g}/L o, in presenza di controindicazioni al test di ipoglicemia insulinica, con picco
inadeguato di GH dopo stimoli alternativi, nei casi di:
\begin{enumerate}
	\renewcommand{\theenumi}{\alph{enumi})}
	\renewcommand{\labelenumi}{\theenumi}
	\item 	ipofisectomia totale o parziale (chirurgica, da radiazioni);
	\item	ipopituitarismo idiopatico, post traumatico, da neoplasie sellari e parasellari.
\end{enumerate}

\subsection*{Background}

\subsubsection*{Et\`a evolutiva}

In soggetti con caratteristiche clinico-auxologiche in accordo con il punto I e con normale secrezione di GH
(punto II), la terapia può essere effettuata solo se autorizzata dalla Commissione Regionale preposta alla
sorveglianza epidemiologica ed al monitoraggio dell'appropriatezza del trattamento con GH in base alle più
recenti acquisizioni scientifiche in materia. 

Per il monitoraggio della prescrizione \`e necessario far riferimento
alla prevalenza del trattamento nella popolazione compresa tra 0 e 18 anni d'et\`a, che \`e stimabile, in base ai
dati della letteratura scientifica internazionale degli ultimi 20 anni, in 1:2000 (tasso di esposizione al
trattamento). Va, inoltre, tenuto conto che la coorte dei soggetti affetti dalle principali patologie per cui \`e
indicata la terapia con GH \`e sostanzialmente stabile nel tempo e distribuita in modo omogeneo sul territorio.

\subsubsection*{Et\`a adulta}

Soggetti adulti con deficit di GH presentano un abbassamento della qualit\`a di vita, una riduzione della forza
muscolare, un aumento dell'adipe viscerale che, insieme ad un aumento del colesterolo circolante, costituisce
un fattore di rischio per complicanze cardiovascolari. In particolare, \`e stato dimostrato un chiaro aumento dei
processi di aterosclerosi con netto incremento della mortalit\`a da cause cardiovascolari. Il trattamento
sostitutivo con GH biosintetico va comunque riservato solo ai rari casi nei quali vi sia un severo deficit di GH,
dimostrato da un picco di risposta $<$ 3  \unit{\micro g}/L dopo ipoglicemia insulinica, oppure, in presenza di
controindicazioni al test dell'ipoglicemia (cardiopatie, patologia del SNC, et\`a avanzata), a seguito di un picco
inadeguato di GH dopo stimoli alternativi utilizzati con limiti di normalit\`a appropriati alla loro potenza. 

Il test
con GHRH + arginina viene ad oggi ritenuto l'alternativa di prima scelta e, dopo questo stimolo, un severo
deficit di GH \`e dimostrato da un picco dei livelli circolanti di GH $<$ 9 \unit{\micro g}/L. Il rigoroso rispetto di tali criteri
esclude la possibilit\`a di un uso improprio o eccessivo del farmaco.

\subsubsection*{Evidenze disponibili}

In base alle evidenze della letteratura, alla approvazione da parte di FDA ed EMEA ed alla luce del documento
della consensus conference ad hoc (J Clin Endocrinol Metab 92: 804–810, 2007), si ritiene opportuno inserire
in fascia A l'indicazione al trattamento con GH nei bambini nati SGA sotto supervisione delle Commissioni
Regionali preposte alla sorveglianza epidemiologica ed al monitoraggio dell'appropriatezza del trattamento
con GH.

\subsubsection*{Particolari avvertenze}

L'Istituto Superiore di Sanit\`a si fa carico della sorveglianza epidemiologica nazionale mediante un Registro
informatizzato dell'ormone della crescita (GH) in collaborazione con le Commissioni Regionali identificate
dalle singole Regioni. Tali commissioni sono preposte alla sorveglianza epidemiologica ed al monitoraggio
dell'appropriatezza del trattamento con GH a livello locale e a tal fine hanno libero accesso ai dati relativi ai
residenti della propria regione. La compilazione del Registro da parte dei Centri prescrittori è necessaria ed indispensabilie per la rimborsabilit\`a della
terapia da parte del SSN. Annualmente l'Istituto Superiore di Sanit\`a provveder\`a a redigere un rapporto e ad
inviarlo all'Agenzia Italiana del Farmaco (AIFA) e alla Conferenza degli Assessori Regionali alla Sanit\`a.
