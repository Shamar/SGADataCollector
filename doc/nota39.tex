\setcounter{page}{46}
\chapter[Nota 39 AIFA - 18 Novembre 2010]
	{Nota 39 AIFA\\[.5ex]
	\normalsize\textit{18 Novembre 2010, GAZZETTA UFFICIALE DELLA REPUBBLICA ITALIANA Serie generale - n. 270}}

[...]

\subsection*{Altre condizioni dove il trattamento con rGH viene concesso in et\`a pediatrica:}

\begin{itemize}
\item sindrome di Turner citogeneticamente dimostrata;
\item deficit staturale nell'insufficienza renale cronica;
\item soggetti prepuberi affetti dalla sindrome di Prader Willi (PWS), geneticamente dimostrata, con Indice di Massa Corporea o Body Mass
Index (BMI) $<$ 95°, normale funzionalit\`a respiratoria, non affetti da sindrome dell'apnea ostruttiva nel
sonno.
\item Bambini nati piccoli per l'et\`a gestazionale (SGA - Small for Gestational Age) con et\`a uguale o superiore a
4 anni.
\end{itemize}

Per poter accedere al trattamento con GH in individui nati SGA \`e necessario rispondere ai seguenti criteri:
\begin{itemize}
\item peso alla nascita nei nati singoli uguale o inferiore a –2 DS ($<$ 3° centile) per l'et\`a gestazionale, basato
sulle tabelle di Gagliardi (L. Gagliardi et Al. “Standard antropometrici neonatali prodotti dalla task-
force della Societ\`a Italiana di Neonatologia e basati su una popolazione italiana nord-orientale” Riv.
Ital. Pediatr. (IJP) 1999; 25: 159-169) e comunque inferiore a 2500 gr.
\item et\`a al momento della proposta di somministrazione del GH uguale o superiore ai 4 anni;
\item statura inferiore o uguale a –2.5 DS e velocit\`a di crescita inferiore al 50° centile.
\item Autorizzazione delle Commissione Regionale preposte alla sorveglianza epidemiologica ed al
monitoraggio dell'appropriatezza del trattamento con GH
\end{itemize}

Considerando la relativa limitata esperienza del trattamento con GH negli SGA in Italia, l'autorizzazione al
trattamento con rGH in soggetti SGA \`e concessa per 2 anni previa verifica ed autorizzazione da parte delle
Commissioni Regionali preposte alla sorveglianza epidemiologica ed al monitoraggio dell'appropriatezza del
trattamento con GH appartenenti alla residenza del paziente. Dopo 2 anni di terapia, il proseguimento
terapeutico potr\`a essere nuovamente autorizzato dalle Commissioni Regionali dopo una verifica dei risultati
clinici ottenuti nel periodo di trattamento.

In caso di mancata istituzione della commissione regionale, la proposta al trattamento con GH da parte del
centro prescrittore dovr\`a essere indirizzata alla Commissione preposta alla sorveglianza epidemiologica ed al
monitoraggio dell'appropriatezza del trattamento con GH presso l'Istituto Superiore di Sanit\`a, che dovr\`a dare
una risposta al centro prescrittore entro giorni trenta dal ricevimento della richiesta.

[...]