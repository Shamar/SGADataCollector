\chapter{Risultati}

La tabella 4.1 riassume la situazione auxologica dei pazienti all'inizio della terapia ed il loro potenziale genetico. La statura è espressa in centimetri e in Standard Deviation Score; il Mid-Parental Height (MPH) è stato calcolato secondo la formula illustrata precedentemente nella sezione Materiali e metodi. Non disponiamo della statura dei genitori delle pazienti B.M. e P.C. perché sono state adottate. Tutti i pazienti arruolati presentavano una statura inferiore alla norma, con deficit staturale medio di -2,8 SDS e range da -4,8 a -1,5 SDS. Le stature dei genitori erano anch'esse piuttosto basse: la statura media dei padri era di 164,9 cm con un inetrvallo da 144,5 a 180,0 cm; la statura media della madri era di 151,5 cm e range da 137,2 a 177,1; il Mid-Parental Height (MPH) medio era ai limiti inferiori della norma, pari a -1,6 SDS, con un range da -3,9 a 1,2 SDS.

\begin{table}[!h]
\begin{center}
\begin{tabular}{lccccccc}
\toprule
 & \multirow{2}{*}{Sesso} & 
   \multirow{2}{*}{Padre} & 
   \multirow{2}{*}{Madre} & 
   \multicolumn{2}{c}{MPH} &
   \multicolumn{2}{c}{Statura} \\
 \cmidrule(r){5-6} \cmidrule(r){7-8}
 & & \emph{cm} & \emph{cm} & \emph{cm} & \emph{SDS}  &		\emph{cm}	& \emph{SDS}             	 \\

\midrule
A.A.	& F & 170,6 & 160,4 & 159,0 & -0,5 	 		& 123,3 & -2,6  \\
B.A.	& F & 170,1 & 156,8 & 156,9 & -0,9   			& 103,7 & -2,7   \\
B.M.	& F & --    & --    & --    & --     			&  97,6 & -3,7  \\
C.D.	& M & 157,7 & 147,8 & 159,2 & -2,3 	        	& 116,1 & -3,4  \\
C.B.	& F & 165,7 & 161,2 & 156,9 & -0,9 	 	        & 125,9 & -2,5  \\
D.V.	& F & 159,2 & 149,8 & 148,0 & -2,4   			& 109,5 & -2,5   \\
D.A.	& M & 157,0 & 143,0 & 156,5 & -2,7 	  		& 120,2 & -2,7  \\
D.S.	& F & 154,8 & 155,9 & 148,8 & -2,3   			& 122,8 & -3,6  \\
E.A.	& M & 144,5 & 139,2 & 148,3 & -3,9 	  		& 113,7 & -4,6  \\
G.A.	& F & 168,2 & 137,2 & 146,2 & -2,7 	  		&  88,1 & -3,2  \\
L.A.	& M & 160,4 & 147,1 & 160,2 & -2,1   			& 128,2 & -3    \\
L.R.	& M & 163,3 & 145,6 & 160,9 & -2,0 	  		& 117,7 & -2,9  \\
L.L.	& M & 164,0 & 143,0 & 160,0 & -2,2 	  		& 136,8 & -2     \\
M.V.	& F & 165,5 & 164,2 & 158,3 & -0,6   			& 135,5 & -2,7  \\
M.E.	& F & 163,2 & 142,6 & 146,4 & -2,7 	  		& 122,6 & -4,8  \\
P.G.	& F & 160,6 & 159,4 & 153,5 & -1,5   			& 121,7 & -2,9  \\
P.D.	& M & 174,2 & 151,8 & 169,5 & -0,8 	  		& 121,5 & -2,3  \\
P.C.	& F & --    & --    & --    & --   	  		& 105,8 & -2,8   \\
P.S.	& F & 180,0 & 171,0 & 169,0 & 1,2  	  		& 132,5 & -3    \\
R.M.	& M & 174,5 & 177,1 & 182,3 & 1,1    			& 112,2 & -3,1  \\
S.M.	& F & 172,8 & 143,1 & 151,4 & -1,8 	        	& 122,0 & -2,2  \\
S.F.	& F & 158,0 & 139,8 & 142,4 & -3,4 	  		& 111,6 & -2,8  \\
S.A.	& F & 174,7 & 153,5 & 157,6 & -0,8 	  		& 120,8 & -2,8  \\
T.G.    & M & --    & --    & 171,5 & -0,5    			& 136,9 & -3,0  \\
V.D.	& M & 171,2 & 146,3 & 165,2 & -1,4 	  		& 149,8 & -1,5  \\
Z.G.	& M & 165,5 & 150,0 & 164,2 & -1,7 	  		& 156,0 & -2     \\
Z.M.	& M & 166,0 & 156,8 & 167,9 & -1,0 	  		& 116,9 & -2,2  \\
Z.L.	& M & 161,4 & 146,1 & 160,2 & -2,1 	  		& 120,1 & -2,8  \\
\bottomrule
\end{tabular}
\end{center}
\caption{Potenziale genetico e situazione auxologica iniziale.}
\label{tab:SituazioneIniziale}
\end{table}

La tabella 4.2 riporta i dati della terapia con ormone della crescita ricombinante. 
Otto pazienti deficitari sono considerati insieme agli altri perchè la risposta alla terapia con rGH non sembra essere correlata allo status secretorio\cite{lee2003international} \cite{de2005growth}.  L'età media a cui i nostri pazienti hanno cominciato la terapia è stata di dieci anni, con un ampia variabilità: la paziente G.A. ha iniziato all'età di quattro anni con un grave deficit staturale; i pazienti V.D. e Z.G. hanno iniziato la terapia ad un'età tardiva (13,9 e 15 anni rispettivamente) a pubertà già iniziata. Questi dati non si discostano da quelli di altre casistiche\cite{coutant1998short} \cite{zucchini2001final}.
In particolare in uno studio svedese sui bambini SGA condotto da Dahlgren e pubblicato nel 2005 l'età media all'inizio del trattamento era di 10,7 anni\cite{dahlgren2005final}.
La durata media media della terapia è stata di cinque anni, con estrema variabilità dovuta al fatto che alcuni pazienti hanno iniziato la terapia a pubertà avanzata. Infatti la durata media del trattamento prepuberale è stata di soli 1,9 anni, anche in questo caso con un grande margine di variabilità: alcuni pazienti hanno potuto beneficiare di un lungo periodo di terapia nell'età infantile (superiore ai sei anni per B.A. e B.M.), altri (A.A., C.B., P.G., V.D., Z.G.) hanno iniziato la terapia a pubertà avanzata. 
In media ai pazienti è stata somministrata una dose pari a 0,27 mg/kg/sett, con una discreta variabilità: la paziente P.S. ha assunto una dose media di 0,21 mg/kg/sett, mentre il paziente P.D. ha beneficiato di una dose piuttosto elevata, pari a 0,36 mg/kg/sett. 
  
\begin{table}[!h]
\begin{center}
%\addtolength{\tabcolsep}{-1pt}
%\renewcommand{\arraystretch}{1.1}
\begin{tabular}{lccrcccl}
\toprule
 &    \multirow{2}{*}{Sesso} &
 \multirow{2}{*}{Deficit}    & \multirow{2}{*}{Età inizio} & \multirow{2}{*}{Dose media} & \multicolumn{2}{c}{Durata Terapia}\\
\cmidrule(r){6-7}
 &       &   &    &  	& Prepuberale & Totale \\
 &     &	&			& \emph{mg/kg/sett}	& \emph{aa} & \emph{aa} \\
\midrule                                  	
A.A.	& F &  		 & 10,6  	&  0,28     	& 0,0                 & 3,6  \\
B.A.	& F & \checkmark & 6,6   	&  0,26     	& 6,1                 & 8,1  \\
B.M.	& F & \checkmark & 6,4   	&  0,30     	& 6,9                 & 7,6  \\
C.D.	& M &  	  	 & 9,7   	&  0,25     	& 3,0                 & 7,1  \\
C.B.	& F &  		 & 10,7  	&  0,25     	& 0,0                 & 3,5  \\
D.V.	& F & \checkmark & 7,5   	&  0,27     	& 3,6                 & 6,6  \\
D.A.	& M &  		 & 9,9   	&  0,28     	& 0,5                 & 5,8  \\
D.S.	& F & \checkmark & 11,5  	&  0,31     	& 0,5                 & 4,0  \\
E.A.	& M &  		 & 10,8  	&  0,30     	& 2,4                 & 6,4  \\
G.A.	& F &  		 & 4,0   	&  0,26     	& 5,5                 & 8,6  \\
L.A.	& M & \checkmark & 12,0  	&  0,26     	& 1,5                 & 3,5  \\
L.R.	& M &  		 & 9,6   	&  0,23     	& 1,5                 & 5,5  \\
L.L.	& M &  		 & 12,7  	&  0,26     	& 1,5                 & 4,5  \\
M.V.	& F & \checkmark & 12,3  	&  0,26     	& 0,5                 & 3,6  \\
M.E.	& F &  		 & 11,6  	&  0,33     	& 0,5                 & 4,0  \\
P.G.	& F & \checkmark & 10,3  	&  0,26     	& 0,0                 & 4,6  \\
P.D.	& M &  		 & 9,6   	&  0,36     	& 1,9                 & 5,9  \\
P.C.	& F &  		 & 7,5   	&  0,31     	& 2,4                 & 6,2  \\
P.S.	& F &  		 & 12,0  	&  0,21     	& 1,0                 & 3,0  \\
R.M.	& M & \checkmark & 6,9   	&  0,29     	& 3,0                 & 8,9  \\
S.M.	& F &  	  	 & 9,7   	&  0,27     	& 0,5                 & 4,7  \\
S.F.	& F &  		 & 8,3   	&  0,26     	& 1,5                 & 6,0  \\
S.A.	& F &  		 & 10,1  	&  0,26     	& 2,0                 & 5,3  \\
T.G.    & M &            & 13,5         &  0,22         & 1,5                 & 2,4  \\
V.D.	& M &  		 & 13,9  	&  0,29     	& 0,0                 & 1,3  \\
Z.G.	& M &  		 & 15,0  	&  0,26     	& 0,0                 & 1,5  \\
Z.M.	& M &  		 & 8,5   	&  0,34     	& 4,5                 & 4,9  \\
Z.L.	& M &  		 & 9,9   	&  0,28     	& 2,0                 & 5,5  \\
\bottomrule
\end{tabular}
\end{center}
\caption{Terapia con GH ricombinante.}
\label{tab:Terapia}
\end{table}

Nella tabella 4.3 sono elencate le variabili relative alla crescita puberale. Del paziente Z.G. non sono note l'età e l'altezza all'inizio della pubertà
La pubertà è iniziata ad un'età normale nelle femmine e nei maschi, rispettivamente a 11,3 (range da 9, a 13,3) e a 12,4 (minimo 9,9, massimo 15,0) anni. 
Alcuni pazienti hanno presentato anticipo puberale fisiologico, nessuno una pubertà precoce vera. 
Sette pazienti (cinque femmine e due maschi) hanno effettuato la terapia antipuberale con  l'analogo sintetico del LHRH (triptorelina alla dose di 100 microg/kg ogni 21-28 giorni, per via intramuscolare) perché presentavano un modesto anticipo puberale e soprattutto perché all'inizio dello sviluppo avevano una statura molto bassa. Le pazienti G.A. e S.F. hanno rifiutato il trattamento con LHRH--analogo.
Le stature a inizio pubertà erano basse per l'età, così comportando un alto rischio di bassa statura definitiva.
In media i pazienti hanno presentato uno scatto puberale soddisfacente: con lo sviluppo i maschi hanno acquisito 26,8 cm; mentre le femmine sono cresciute in media 20,5 cm. 
%Per confronto, nella poplazione generale lo scatto di crescita puberale medio è pari a 23 cm per i maschi e 20 cm per le femmine.  \cite{tanner1990foetus} questo confronto va nella discussione
I nostri dati hanno confermato quanto già verificato da altri autori.\cite{boonstra2003puberty} %citare n 68 consensus
 
\begin{table}[!h]
\begin{center}
%\addtolength{\tabcolsep}{12pt}
%\renewcommand{\arraystretch}{1.1}
\begin{tabular}{lccccc}
\toprule
 & Sesso 	& \multicolumn{1}{c}{Età inizio}	& Statura	& Terapia frenante & Scatto puberale \\
 & &  \emph{aa} 	& \emph{cm}	& & \emph{cm}		\\
\midrule
A.A.	& F & 10,2 		& 121,8  		& \checkmark & 24,2 \\
B.A.	& F & 12,7 		& 137,9   		&            & 11,1 \\
B.M.	& F & 13,3 		& 141,5   		&            & 19,3 \\
C.D.	& M & 12,8 		& 136,2   		&            & 26,7 \\
C.B.	& F & 10,5 		& 125,0   		& \checkmark & 30,6 \\
D.V.	& F & 11,1 		& 131,0   		&            & 15,0 \\
D.A.	& M & 10,4 		& 123,6   		& \checkmark & 35,4 \\
D.S.	& F & 12,0 		& 126,3   		&            & 18,4 \\
E.A.	& M & 13,3		& 125,7   		&            & 21,6 \\
G.A.	& F &  9,5  		& 122,5   		&            & 20,5 \\
L.A.	& M & 13,5 		& 134,3   		&            & 25,9 \\
L.R.	& M & 11,1 		& 127,5   		&            & 33,9 \\
L.L.	& M & 14,2 		& 146,6   		&            & 20,7 \\
M.V.	& F & 12,8 		& 139,3   		&            & 18,2 \\
M.E.	& F & 12,1 		& 126,8   		&            & 20,0 \\
P.G.	& F & 10,0 		& 121,1   		& \checkmark & 26,4 \\
P.D.	& M & 11,5 		& 132,9   		&            & 24,5 \\
P.C.	& F &  9,9  		& 123,8   		& \checkmark & 20,4 \\
P.S.	& F & 13,0 		& 140,0   		&            & 18,0 \\
R.M.	& M &  9,9  		& 135,6   		& \checkmark & 35,8 \\
S.M.	& F & 10,3 		& 126,8   		& \checkmark & 21,2 \\
S.F.	& F &  9,8  		& 124,1   		&            & 23,1 \\
S.A.	& F & 12,0 		& 137,8   		&            & 20,8 \\
T.G.    & M & 15,0      	& 146,5         	&            & 18,0 \\ 
V.D.	& M & 11,7 		& 138,4   		&            & 33,1 \\
Z.G.	& M &  --    		& --   			&            & --   \\
Z.M.	& M & 13,0 		& 139,5   		&            & 23,5 \\
Z.L.	& M & 11,9 		& 132,6   		&            & 22,4 \\
\bottomrule
\end{tabular}
\end{center}
\caption{Crescita puberale.}
\label{tab:Puberta}
\end{table}

Le medie relative alla pubertà sono riassunte nella tabella 4.4.

\begin{table}[!h]
\begin{center}
\addtolength{\tabcolsep}{-2pt}
\begin{tabular}{l c c c clrclr c clrclr}
\toprule
 & & & &\multicolumn{6}{c}{Maschi}& & \multicolumn{6}{c}{Femmine}\\
\midrule
Età Inizio \emph{(aa)}             & & & & 12,4  &  & (9,9    &\div &  15,0) &   & & 11,3  &  &(9,5    &\div &  13,3) &  \\ 
Altezza inizio \emph{(cm)}         & & & & 135,0 &  & (123,6  &\div & 146,6) &   & & 129,8 &  &(121,1  &\div & 141,5) &  \\ 
Scatto puberale \emph{(cm)}        & & & & 26,8  &  & (18,0   &\div &  35,8) &   & & 20,5  &  &(11,1   &\div &  30,6) &  \\ 
\bottomrule
\end{tabular} 
\end{center}
\caption{Medie relative alla pubertà (per sesso).}
\label{tab:MediePerSesso}
\end{table}

La tabella 4.5 presenta l'esito della terapia con ormone somatotropo ricombinante nella nostra casistica.
Per ogni paziente ho calcolato la differenza fra la statura finale espressa in SDS e la statura a inizio terapia, anch'essa in SDS (guadagno staturale in SDS).
Per procedere al confronto della statura raggiunta rispetto al potenziale genetico (MPH) ho calcolato la differenza fra la statura definitiva dei pazienti in SDS ed l'MPH in SDS.
La tabella 4.5 illustra i dati dei singoli pazienti; i valori medi sono riportati nella tabella 4.6. 


\begin{table}[!h]
\begin{center}
%\addtolength{\tabcolsep}{12pt}
%\renewcommand{\arraystretch}{1.1}
\begin{tabular}{lccccc}
\toprule
 & Sesso & \multicolumn{2}{c}{Statura finale} 	& \Delta SDS & SDS dal MPH	 \\
\cmidrule(r){3-4}
  &  & \emph{cm} 	& \emph{SDS}  	   	&		& 			 	\\
\midrule
A.A.	& F & 146,0 & -2,7  & -0,1 & -2,2      \\
B.A.	& F & 149,0 & -2,2  & 0,5  & -1,3      \\
B.M.	& F & 160,8 & -0,2  & 3,5  & --        \\
C.D.	& M & 162,9 & -1,8  & 1,6  & 0,5       \\
C.B.	& F & 155,6 & -1,1  & 1,4  & -0,2      \\
D.V.	& F & 146,0 & -2,7  & -0,2 & -0,3      \\
D.A.	& M & 159,0 & -2,5  & 0,2  & 0,2        \\
D.S.	& F & 144,7 & -3,0  & 0,6  & -0,7      \\
E.A.	& M & 147,3 & -4,0  & 0,6  & -0,1      \\
G.A.	& F & 143,0 & -3,5  & -0,3 & -0,8          \\
L.A.	& M & 160,2 & -2,1  & 0,9  & 0,0       \\
L.R.	& M & 161,4 & -2,0  & 0,9  & 0,0       \\
L.L.	& M & 167,3 & -1,1  & 0,9  & 1,1       \\
M.V.	& F & 157,5 & -0,8  & 1,9  & -0,2      \\
M.E.	& F & 146,8 & -2,6  & 2,2  & 0,1       \\
P.G.	& F & 147,5 & -2,5  & 0,4  & -1,0      \\
P.D.	& M & 157,4 & -2,5  & -0,2 & -1,7      \\
P.C.	& F & 144,2 & -3,0  & -0,2 & --       \\
P.S.	& F & 158,0 & -0,7  & 2,3  & -1,9      \\
R.M.	& M & 171,4 & -0,5  & 2,6  & -1,6     \\
S.M.	& F & 148,0 & -2,4  & -0,2 & -0,6      \\
S.F.	& F & 147,2 & -2,5  & 0,3  & 0,9       \\
S.A.	& F & 158,6 & -0,6  & 2,2  & 0,2       \\
T.G.    & M & 164,5 & -1,6  & 1,4  & -1,1     \\
V.D.	& M & 171,5 & -0,5  & 1,0  & 0,9       \\
Z.G.	& M & 166,3 & -1,3  & 0,7  & 0,4     \\
Z.M.	& M & 163,0 & -1,7  & 0,5  & -0,7      \\
Z.L.	& M & 155,0 & -2,9  & -0,1 & -0,8      \\
\bottomrule
\end{tabular}
\end{center}
\caption{Esito della terapia con rGH.}
\label{tab:Esito}
\end{table}



\begin{table}[!h]
\begin{center}
%\addtolength{\tabcolsep}{-4pt}
\begin{tabular}{l c c c clrclr}
\toprule
                                & & & & \multicolumn{6}{c}{Media} \\ 
\midrule                        
Età inizio terapia  \emph{(aa)}            & & & & 10,0 & &(4,0  &\div & 15,0)& \\ 
MPH      \emph{(SDS)}           & & & & -1,6 & &(-3,9 &\div &  1,2)&  \\ 
Statura inizio terapia \emph{(SDS)}         & & & & -2,9 & &(-4,8 &\div & -1,5)& \\ 
Dose \emph{(mg/kg/sett)}        & & & & 0,27 & &(0,21 &\div & 0,36)& \\ 
Durata della terapia \emph{(aa)}& & & & 5    & &(1,3  &\div &  8,9)&  \\
Anni prepuberali trattati  \emph{(aa)}     & & & & 1,9  & &(0    &\div &  6,9)&  \\ 
Final Height \emph{(SDS)}       & & & & -2,0 & &(-4,0 &\div & -0,2)& \\ 
\Delta SDS                      & & & & 0,9  & &(-0,3 &\div &  3,5)&  \\ 
SDS dal MPH                     & & & & -0,4 & &(-2,2 &\div &  1,1)&  \\ 
\bottomrule
\end{tabular} 
\end{center}
\caption{Medie relative alla terapia e all'esito.}
\label{tab:Medie}
\end{table}

La statura definitiva media corrisponde a -2 SDS. Il risultato include ovviamente bambini cresciuti molto e bambini cresciuti poco (range da -4,0 a -0,2 SDS). Tuttavia il guadagno netto corrisponde a 0,9 SDS, cioè a circa 5 cm per le femmine e 6 cm  nei maschi.

Alcuni pazienti non hanno avuto beneficio dalla terapia: i pazienti A.A., P.D., S.M. e Z.L. hanno inziato il trattamento tardi, sono stati trattati per un periodo di tempo non superiore a 2 anni prima dello sviluppo e quindi hanno iniziato la pubertà ad una bassa statura. In questi pazienti il normale incremento staturale nello scatto puberale (superiore a 20 cm nelle femmine, maggiore di 25 cm nei maschi) non è bastato a normalizzare la statura definitiva. 
  
La paziente G.A. ha iniziato lo sviluppo a 9,5 anni ad una statura di 122,5 cm, ha rifiutato la terapia con LHRH-analogo ed ha presentato uno scatto puberale di 21,5 cm. La sua statura adulta, sebbene deficitaria (-3,5 SDS) è risultata ben compatibile con il suo potenziale genetico (MPH). 

Il paziente E.A. apparteneva ad una famiglia di persone molto basse (MPH = -3,9 SDS), presentava normali livelli di GH e IGF-1 ed ha raggiunto una statura finale estremamente bassa (-4,0 SDS), in quadro di possibile resistenza all'IGF-1. La terapia ha permesso al ragazzo di centrare il bersaglio genetico e superare la statura paterna.

I pazienti D.V. e Z.L. hanno presentato durante la pubertà un incremento staturale inferiore alla media, esitato in statura adulta disarmonica con brevità mesomelica degli arti inferiori. Sono ancora in corso accertamenti per valutare un'eventuale mutazione del gene SHOX. 

  


\begin{table}[!h]
\begin{center}
%\addtolength{\tabcolsep}{-4pt}
\begin{tabular}{l c c c c}
\toprule
\multirow{2}{*}{Modello}& \multirow{2}{*}{Sig.} & \multicolumn{2}{c}{Intervallo di confidenza 95,0\%} \\
			& 	& Limite inferiore & Limite superiore \\
\midrule
MPH			& 0,008	& 0,021	  & 0,117 \\
Sesso			& 0,022	& 0,194	  & 2,205 \\
Deficit GH		& 0,737	& -0,925  & 0,668 \\
Età inizio		& 0,087	& -0,067  & 0,888 \\
Dose media		& 0,596	& -12,386 & 7,368 \\
dt Prepuberale		& 0,730	& -0,560  & 0,401 \\
durataTotale		& 0,058	& -0,017  & 0,921 \\
Terapia Frenante	& 0,388	& -1,665  & 0,684 \\
\bottomrule
\end{tabular} 
\end{center}
\caption{Coefficienti di regressione lineare multivariata rispetto alla variabile dipendente definita dal guadagno in SDS.}
\label{tab:Statistiche}
\end{table}

