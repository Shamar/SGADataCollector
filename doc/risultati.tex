\chapter{Risultati}

La tabella 4.1 riassume la situazione auxologica dei pazienti all'inizio della terapia ed il loro potenziale genetico. La statura è espressa in centimetri e in Standard Deviation Score; il Mid-Parental Height (MPH) è stato calcolato secondo la formula illustrata precedentemente nella sezione Materiali e metodi. Non disponiamo della statura dei genitori delle pazienti B.M. e P.C. perchè sono state adottate. Tutti i pazienti arruolati presentavano una statura inferiore alla norma, con deficit staturale medio di -2,8 SDS e range da -4,8 a -1,5 SDS. Le stature dei genitori erano anch'esse piuttosto basse: la statura media dei padri era di 164,9 cm con un inetrvallo da 144,5 a 180,0 cm; la statura media della madri era di 151,5 cm e range da 137,2 a 177,1; il Mid-Parental Height (MPH) medio era ai limiti inferiori della norma, pari a -1,6 SDS, con un range da -3,9 a 1,2 SDS.

\begin{table}[!h]
\begin{center}
\begin{tabular}{lccccccc}
\toprule
 & \multirow{2}{*}{Sesso} & 
   \multirow{2}{*}{Padre} & 
   \multirow{2}{*}{Madre} & 
   \multirow{2}{*}{MPH} & 
   SDS  & 
   
   \multicolumn{2}{c}{Altezza} \\
 \cmidrule(r){7-8}
 & &   	& & & del MPH &		cm				& \emph{SDS}             	 \\
\midrule
A.A.	& F & 170,6 & 160,4 & 159,0 & -0,5 	 		& 123,3 & -2,6  \\
B.A.	& F & 170,1 & 156,8 & 156,9 & -0,9   			& 103,7 & -2,7   \\
B.M.	& F & --    & --    & --    & --     			&  97,6 & -3,7  \\
C.D.	& M & 157,7 & 147,8 & 159,2 & -2,3 	        	& 116,1 & -3,4  \\
C.B.	& F & 165,7 & 161,2 & 156,9 & -0,9 	 	        & 125,9 & -2,5  \\
D.V.	& F & 159,2 & 149,8 & 148,0 & -2,4   			& 109,5 & -2,5   \\
D.A.	& M & 157,0 & 143,0 & 156,5 & -2,7 	  		& 120,2 & -2,7  \\
D.S.	& F & 154,8 & 155,9 & 148,8 & -2,3   			& 122,8 & -3,6  \\
E.A.	& M & 144,5 & 139,2 & 148,3 & -3,9 	  		& 113,7 & -4,6  \\
G.A.	& F & 168,2 & 137,2 & 146,2 & -2,7 	  		&  88,1 & -3,2  \\
L.A.	& M & 160,4 & 147,1 & 160,2 & -2,1   			& 128,2 & -3    \\
L.R.	& M & 163,3 & 145,6 & 160,9 & -2,0 	  		& 117,7 & -2,9  \\
L.L.	& M & 164,0 & 143,0 & 160,0 & -2,2 	  		& 136,8 & -2     \\
M.V.	& F & 165,5 & 164,2 & 158,3 & -0,6   			& 135,5 & -2,7  \\
M.E.	& F & 163,2 & 142,6 & 146,4 & -2,7 	  		& 122,6 & -4,8  \\
P.G.	& F & 160,6 & 159,4 & 153,5 & -1,5   			& 121,7 & -2,9  \\
P.D.	& M & 174,2 & 151,8 & 169,5 & -0,8 	  		& 121,5 & -2,3  \\
P.C.	& F & --    & --    & --    & --   	  		& 105,8 & -2,8   \\
P.S.	& F & 180,0 & 171,0 & 169,0 & 1,2  	  		& 132,5 & -3    \\
R.M.	& M & 174,5 & 177,1 & 182,3 & 1,1    			& 112,2 & -3,1  \\
S.M.	& F & 172,8 & 143,1 & 151,4 & -1,8 	        	& 122,0 & -2,2  \\
S.F.	& F & 158,0 & 139,8 & 142,4 & -3,4 	  		& 111,6 & -2,8  \\
S.A.	& F & 174,7 & 153,5 & 157,6 & -0,8 	  		& 120,8 & -2,8  \\
T.G.    & M & --    & --    & 171,5 & -0,5    			& 136,9 & -3,0  \\
V.D.	& M & 171,2 & 146,3 & 165,2 & -1,4 	  		& 149,8 & -1,5  \\
Z.G.	& M & 165,5 & 150,0 & 164,2 & -1,7 	  		& 156,0 & -2     \\
Z.M.	& M & 166,0 & 156,8 & 167,9 & -1,0 	  		& 116,9 & -2,2  \\
Z.L.	& M & 161,4 & 146,1 & 160,2 & -2,1 	  		& 120,1 & -2,8  \\
\bottomrule
\end{tabular}
\end{center}
\caption{Potenziale genetico e situazione auxologica iniziale.}
\label{tab:SituazioneIniziale}
\end{table}

L'età media a cui i nostri pazienti hanno cominciato la terapia è stata di dieci anni, con un ampia variabilità: la paziente G.A. ha iniziato all'età di quattro anni con un grave deficit staturale; i pazienti V.D. e Z.G. hanno iniziato la terapia ad un'età tardiva (13,9 e 15 anni rispettivamente) a pubertà già iniziata. Questi dati non si discostano da quelli delle casistiche dell'Europa settentrionale: in uno studio svedese sui bambini SGA in terapia con rGH pubblicato nel 2005 l'età media all'inizio del trattamento era di 10,7 anni.%citare Dahlgren (ho salvato la pag web)
In media la terapia è durata complessivamente cinque anni, con un range estremamente variabile da 1,3 a 8,9 anni. La durata media del trattamento nel periodo precedente lo sviluppo puberale è stata di soli 1,9 anni, anche in questo caso con un grande margine di variabilità: alcuni pazienti hanno potuto beneficiare di un lungo periodo di terapia nell'età infantile (circa sette anni), altri hanno iniziato la terapia a pubertà avanzata. In media ai pazienti è stata somministrata una dose pari a 0,27 mg/kg/sett, con un'ampia variabilità: la paziente P.S. ha assunto una dose media di 0,21 mg/kg/sett, mentre il paziente P.D. ha beneficiato di una dose piuttosto elevata, pari a 0,36 mg/kg/sett. 
  
\begin{table}[!h]
\begin{center}
%\addtolength{\tabcolsep}{-1pt}
%\renewcommand{\arraystretch}{1.1}
\begin{tabular}{lccrcccl}
\toprule
 &    \multirow{2}{*}{Sesso} &
 \multirow{2}{*}{Deficit}    & \multirow{2}{*}{Età inizio} & \multirow{2}{*}{Dose media} & \multicolumn{2}{c}{Durata Terapia}\\
\cmidrule(r){6-7}
 &       &   &    &  	& Prepuberale & Totale \\
 &     &	&			& \emph{mg/kg/sett}	& \emph{aa} & \emph{aa} \\
\midrule                                  	
A.A.	& F &  		 & 10,6  	&  0,28     	& 0,0                 & 3,6  \\
B.A.	& F & \checkmark & 6,6   	&  0,26     	& 6,1                 & 8,1  \\
B.M.	& F & \checkmark & 6,4   	&  0,30     	& 6,9                 & 7,6  \\
C.D.	& M &  	  	 & 9,7   	&  0,25     	& 3,0                 & 7,1  \\
C.B.	& F &  		 & 10,7  	&  0,25     	& 0,0                 & 3,5  \\
D.V.	& F & \checkmark & 7,5   	&  0,27     	& 3,6                 & 6,6  \\
D.A.	& M &  		 & 9,9   	&  0,28     	& 0,5                 & 5,8  \\
D.S.	& F & \checkmark & 11,5  	&  0,31     	& 0,5                 & 4,0  \\
E.A.	& M &  		 & 10,8  	&  0,30     	& 2,4                 & 6,4  \\
G.A.	& F &  		 & 4,0   	&  0,26     	& 5,5                 & 8,6  \\
L.A.	& M & \checkmark & 12,0  	&  0,26     	& 1,5                 & 3,5  \\
L.R.	& M &  		 & 9,6   	&  0,23     	& 1,5                 & 5,5  \\
L.L.	& M &  		 & 12,7  	&  0,26     	& 1,5                 & 4,5  \\
M.V.	& F & \checkmark & 12,3  	&  0,26     	& 0,5                 & 3,6  \\
M.E.	& F &  		 & 11,6  	&  0,33     	& 0,5                 & 4,0  \\
P.G.	& F & \checkmark & 10,3  	&  0,26     	& 0,0                 & 4,6  \\
P.D.	& M &  		 & 9,6   	&  0,36     	& 1,9                 & 5,9  \\
P.C.	& F &  		 & 7,5   	&  0,31     	& 2,4                 & 6,2  \\
P.S.	& F &  		 & 12,0  	&  0,21     	& 1,0                 & 3,0  \\
R.M.	& M & \checkmark & 6,9   	&  0,29     	& 3,0                 & 8,9  \\
S.M.	& F &  	  	 & 9,7   	&  0,27     	& 0,5                 & 4,7  \\
S.F.	& F &  		 & 8,3   	&  0,26     	& 1,5                 & 6,0  \\
S.A.	& F &  		 & 10,1  	&  0,26     	& 2,0                 & 5,3  \\
T.G.    & M &            & 13,5         &  0,22         & 1,5                 & 2,4  \\
V.D.	& M &  		 & 13,9  	&  0,29     	& 0,0                 & 1,3  \\
Z.G.	& M &  		 & 15,0  	&  0,26     	& 0,0                 & 1,5  \\
Z.M.	& M &  		 & 8,5   	&  0,34     	& 4,5                 & 4,9  \\
Z.L.	& M &  		 & 9,9   	&  0,28     	& 2,0                 & 5,5  \\
\bottomrule
\end{tabular}
\end{center}
\caption{Terapia.}
\label{tab:Terapia}
\end{table}

La pubertà nei maschi è iniziata ad un'età media di 12,4 anni, con una variabilità piuttosto ampia: il paziente R.M. ha iniziato lo sviluppo a circa dieci anni, mentre il paziente T.G. all'età di quindici anni. Nelle femmine la pubertà è iniziata in media a 11,3 anni, con un range da 9,5 anni a 13,3 anni. Nessuno dei pazienti in esame ha presentato una pubertà precoce vera. Sette pazienti (cinque femmine e due maschi) hanno effettuato la terapia antipuberale con un analogo sintetico del LHRH perchè presentavano un modesto anticipo puberale e soprattutto perchè all'inizio dello sviluppo avevano una statura molto bassa.
 
\begin{table}[!h]
\begin{center}
%\addtolength{\tabcolsep}{12pt}
%\renewcommand{\arraystretch}{1.1}
\begin{tabular}{lcccc}
\toprule
 & Sesso 	& \multicolumn{1}{c}{Età inizio}	& Altezza	& Terapia frenante \\
 & &  \multicolumn{1}{c}{\emph{aa}} 	& \multicolumn{1}{c}{\emph{cm}}			\\
\midrule
A.A.	& F & 10,2 		& 121,8  		& \checkmark \\
B.A.	& F & 12,7 		& 137,9   		&            \\
B.M.	& F & 13,3 		& 141,5   		&            \\
C.D.	& M & 12,8 		& 136,2   		&            \\
C.B.	& F & 10,5 		& 125,0   		& \checkmark \\
D.V.	& F & 11,1 		& 131,0   		&            \\
D.A.	& M & 10,4 		& 123,6   		& \checkmark \\
D.S.	& F & 12,0 		& 126,3   		&            \\
E.A.	& M & 13,3		& 125,7   		&            \\
G.A.	& F &  9,5  	& 122,5   		&            \\
L.A.	& M & 13,5 		& 134,3   		&            \\
L.R.	& M & 11,1 		& 127,5   		&            \\
L.L.	& M & 14,2 		& 146,6   		&            \\
M.V.	& F & 12,8 		& 139,3   		&            \\
M.E.	& F & 12,1 		& 126,8   		&            \\
P.G.	& F & 10,0 		& 121,1   		& \checkmark \\
P.D.	& M & 11,5 		& 132,9   		&            \\
P.C.	& F &  9,9  		& 123,8   		& \checkmark \\
P.S.	& F & 13,0 		& 140,0   		&            \\
R.M.	& M &  9,9  		& 135,6   		& \checkmark \\
S.M.	& F & 10,3 		& 126,8   		& \checkmark \\
S.F.	& F &  9,8  		& 124,1   		&            \\
S.A.	& F & 12,0 		& 137,8   		&            \\
T.G.    & M & 15,0      & 146,5         &            \\ 
V.D.	& M & 11,7 		& 138,4   		&            \\
Z.G.	& M &  --    		& --   		&            \\
Z.M.	& M & 13,0 		& 139,5   		&            \\
Z.L.	& M & 11,9 		& 132,6   		&            \\
\bottomrule
\end{tabular}
\end{center}
\caption{Pubertà.}
\label{tab:Puberta}
\end{table}

In media i pazienti hanno effettuato un buono scatto puberale: mediamente i maschi hanno acquisito 26,8 cm con lo sviluppo, con un range da 18,0 a 35,8 cm; mentre le femmine sono cresciute in media 20,5 cm, con un intervallo da 11,1 a 30,5 cm . Per confronto, nella poplazione generale lo scatto di crescita puberale medio è pari a 23 cm per i maschi e 20 cm per le femmine.  %citare foetus into man  di tanner(è già in bibliografia)
 Per quanto concerne la statura finale, in media è stata raggiunta una statura ai limiti inferiori della curva di normalutà per la poplazione generale, anche qui con un ampio margine di variabilità: il paziente E.A. non ha normalizzato la sua statura definitiva, che è risultata inferiore di 4 SDS rispetto alla media della popolazione generale; sei pazienti hanno raggiunto una statura finale che si discosta meno di una SDS dalla statura media della popolazione generale. 
Se confrontiamo la statura definitiva dei pazienti con il loro MPH, cioè con la statura più probabile per il loro potenziale genetico, possiamo osservare che in media i pazienti hanno pressochè raggiunto il centro della fascia-bersaglio parentale (MPH). Solamente una paziente ha raggiunto una statura definitiva al limite inferiore del proprio potenziale genetico (-2,2 SDS dal MPH), mentre nove pazienti hanno superato il loro MPH. 
In media i pazienti hanno guadagnato circa una SDS rispetto alla stutura all'inizio del trattamento, con un'ampia variabilità: la paziente G.A. ha riportato un lieve peggioramento, mentre la paziente B.M. ha presenato un guadagno importante pari a 3,5 SDS.
  
\begin{table}[!h]
\begin{center}
%\addtolength{\tabcolsep}{12pt}
%\renewcommand{\arraystretch}{1.1}
\begin{tabular}{lcccccc}
\toprule
 & Sesso  & Scatto puberale & \multicolumn{2}{c}{Altezza finale} 	& SDS dal MPH	& \Delta SDS \\
\cmidrule(r){4-5}
  & & \emph{cm} & \emph{cm} 	& \emph{SDS}  	   	&		& 			 	\\
\midrule
A.A.	& F & 24,2 & 146,0 & -2,7  & -2,2  & -0,1    \\
B.A.	& F & 11,1 & 149,0 & -2,2  & -1,4  & 0,5     \\
B.M.	& F & 19,3 & 160,8 & -0,2  & --    & 3,5     \\
C.D.	& M & 26,7 & 162,9 & -1,8  & 0,5   & 1,6     \\
C.B.	& F & 30,6 & 155,6 & -1,1  & -0,2  & 1,4     \\
D.V.	& F & 15,0 & 146,0 & -2,7  & -0,3  & -0,2    \\
D.A.	& M & 35,4 & 159,0 & -2,5  & 0,2   & 0,2      \\
D.S.	& F & 18,4 & 144,7 & -3,0  & -0,7  & 0,6     \\
E.A.	& M & 21,6 & 147,3 & -4,0  & -0,1  & 0,6     \\
G.A.	& F & 20,5 & 143,0 & -3,5  & -0,8  & -0,3        \\
L.A.	& M & 25,9 & 160,2 & -2,1  & 0,0   & 0,9     \\
L.R.	& M & 33,9 & 161,4 & -2,0  & 0,1   & 0,9     \\
L.L.	& M & 20,7 & 167,3 & -1,1  & 1,1   & 0,9     \\
M.V.	& F & 18,2 & 157,5 & -0,8  & -0,1  & 1,9     \\
M.E.	& F & 20,0 & 146,8 & -2,6  & 0,1   & 2,2     \\
P.G.	& F & 26,4 & 147,5 & -2,5  & -1,0  & 0,4     \\
P.D.	& M & 24,5 & 157,4 & -2,5  & -1,8  & -0,2    \\
P.C.	& F & 20,4 & 144,2 & -3,0  & --  & -0,2   \\
P.S.	& F & 18,0 & 158,0 & -0,7  & -1,9  & 2,3     \\
R.M.	& M & 35,8 & 171,4 & -0,5  & -1,6  & 2,6    \\
S.M.	& F & 21,2 & 148,0 & -2,4  & -0,6  & -0,2    \\
S.F.	& F & 23,1 & 147,2 & -2,5  & 0,8   & 0,3     \\
S.A.	& F & 20,8 & 158,6 & -0,6  & 0,2   & 2,2     \\
T.G.    & M & 18,0 & 164,5 & -1,6  & -1,1  & 1,4    \\
V.D.	& M & 33,1 & 171,5 & -0,5  & 0,9   & 1,0     \\
Z.G.	& M & --   & 166,3 & -1,3  & 0,4   & 0,7   \\
Z.M.	& M & 23,5 & 163,0 & -1,7  & -0,7  & 0,5     \\
Z.L.	& M & 22,4 & 155,0 & -2,9  & -0,8  & -0,1    \\
\bottomrule
\end{tabular}
\end{center}
\caption{Esito.}
\label{tab:Esito}
\end{table}


\begin{table}[!h]
\begin{center}
%\addtolength{\tabcolsep}{-4pt}
\begin{tabular}{l c c c clrclr}
\toprule
                                & & & & \multicolumn{6}{c}{Media} \\ 
\midrule                        
Statura padre\emph{(cm)}        & & & & 164,9 & & (144,5\div & 180,0)& \\
Statura madre\emph{(cm)}        & & & & 151,5 & & (137,2\div & 177,1)& \\
MPH      \emph{(SDS)}           & & & & -1,6 & &(-3,9 &\div &  1,2)&  \\ 
Statura inizio terapia \emph{(SDS)}         & & & & -2,8 & &(-4,8 &\div & -1,5)& \\ 
Età inizio terapia  \emph{(aa)}            & & & & 10,0 & &(4,0  &\div & 15,0)& \\ 
Dose \emph{(mg/kg/sett)}        & & & & 0,27 & &(0,21 &\div & 0,36)& \\ 
Anni prepuberali trattati  \emph{(aa)}     & & & & 1,9  & &(0    &\div &  6,9)&  \\ 
Durata della terapia \emph{(aa)}& & & & 5    & &(1,3  &\div &  8,9)&  \\ 
Final Height \emph{(SDS)}       & & & & -2,0 & &(-4,0 &\div & -0,2)& \\ 
SDS dal MPH                     & & & & -0,4 & &(-2,2 &\div &  1,1)&  \\ 
\Delta SDS                      & & & & 0,9  & &(-0,3 &\div &  3,5)&  \\ 
\bottomrule
\end{tabular} 
\end{center}
\caption{Medie.}
\label{tab:Medie}
\end{table}

\begin{table}[!h]
\begin{center}
\addtolength{\tabcolsep}{-2pt}
\begin{tabular}{l c c c clrclr c clrclr}
\toprule
 & & & &\multicolumn{6}{c}{Maschi}& & \multicolumn{6}{c}{Femmine}\\
\midrule
Età Inizio \emph{(aa)}             & & & & 12,4  &  & (9,9    &\div &  15,0) &   & & 11,3  &  &(9,5    &\div &  13,3) &  \\ 
Altezza inizio \emph{(cm)}         & & & & 135,0 &  & (123,6  &\div & 146,6) &   & & 129,8 &  &(121,1  &\div & 141,5) &  \\ 
Scatto puberale \emph{(cm)}        & & & & 26,8  &  & (18,0   &\div &  35,8) &   & & 20,5  &  &(11,1   &\div &  30,6) &  \\ 
\bottomrule
\end{tabular} 
\end{center}
\caption{Medie relative alla pubertà (per sesso).}
\label{tab:MediePerSesso}
\end{table}


\begin{sidewaystable}
\centering
\begin{tabular}{lccccccccccc}
\toprule
 	& Sesso & Età I.T. & Deficit GH & HSD % altezza inizio trattamento in SDS 
 	& TH % target in SDS (MPH in SDS)
 	& Età I.P.
 	& Dose GH
 	& Durata T.
 	& FHSDS
 	& \Delta SDS
 	& Corr MPH
 	\\
\midrule                                
A.A. & F & 10,6 &  		& -2,6 & -0,5 & 10,2 & 0,28 & 3,6 & -2,7 & -0,1 & -2,2                    \\
B.A. & F & 6,6  & \checkmark 	& -2,7 & -0,9 & 12,7 & 0,26 & 8,1 & -2,2 & 0,5  & -1,4                    \\
B.M. & F & 6,4  & \checkmark 	& -3,7 & --   & 13,3 & 0,30 & 7,6 & -0,2 & 3,5  & --                      \\
C.D. & M & 9,7  &  	            & -3,4 & -2,3 & 12,8 & 0,25 & 7,1 & -1,8 & 1,6  & 0,5                     \\
C.B. & F & 10,7 &  		& -2,5 & -0,9 & 10,5 & 0,25 & 3,5 & -1,1 & 1,4  & -0,2                           \\
D.V.& F & 7,5  & \checkmark 	& -2,5 & -2,4 & 11,1 & 0,27 & 6,6 & -2,7 & -0,2 & -0,3                    \\
D.A. & M & 9,9  &  		& -2,7 & -2,7 & 10,4 & 0,28 & 5,8 & -2,5 & 0,2     & 0,2                               \\
D.S.& F & 11,5 & \checkmark 	& -3,6 & -2,3 & 12,0 & 0,31 & 4,0 & -3,0 & 0,6  & -0,7                    \\
E.A. & M & 10,8 &  		& -4,6 & -3,9 & 13,3 & 0,30 & 6,4 & -4,0 & 0,6  & -0,1                           \\
G.A. & F & 4,0  &  		& -3,2 & -2,7 & 9,5  & 0,26 & 8,6 & -3,5 & -0,3 & -0,8                               \\
L.A. & M & 12,0 & \checkmark 	&  -3  & -2,1 & 13,5 & 0,26 & 3,5 & -2,1 & 0,9  & 0,0                     \\
L.R. & M & 9,6  &  		& -2,9 & -2,0 & 11,1 & 0,23 & 5,5 & -2,0 & 0,9  & 0,1                            \\
L.L. & M & 12,7 &  		&  -2  & -2,2 & 14,2 & 0,26 & 4,5 & -1,1 & 0,9  & 1,1                            \\
M.V. & F & 12,3 & \checkmark 	& -2,7 & -0,6 & 12,8 & 0,26 & 3,6 & -0,8 & 1,9  & -0,1                    \\
M.E. & F & 11,6 &  		& -4,8 & -2,7 & 12,1 & 0,33 & 4,0 & -2,6 & 2,2  & 0,1                            \\
P.G. & F & 10,3 & \checkmark 	& -2,9 & -1,5 & 10,0 & 0,26 & 4,6 & -2,5 & 0,4  & -1,0                    \\
P.D. & M & 9,6  &  		& -2,3 & -0,8 & 11,5 & 0,36 & 5,9 & -2,5 & -0,2 & -1,8                           \\
P.C. & F & 7,5  &  		& -2,8 & --   & 9,9  & 0,31 & 6,2 & -3,0 & -0,2 & --                           \\
P.S. & F & 12,0 &  		&  -3  & 1,2  & 13,0 & 0,21 & 3,0 & -0,7 & 2,3  & -1,9                           \\
R.M. & M & 6,9  & \checkmark 	& -3,1 & 1,1  & 9,9  & 0,29 & 8,9 & -0,5 & 2,6  & -1,6                    \\
S.M. & F & 9,7  &  	       & -2,2 & -1,8 & 10,3 & 0,27 & 4,7 & -2,4 & -0,2 & -0,6                    \\
T.G. & M & 13,5   &             & -3,0 & -0,5 & 15,0 & 0,22 & 2,4 & -1,6 & 1,4 & -1,1     \\
S.F. & F & 8,3  &  		& -2,8 & -3,4 & 9,8  & 0,26 & 6,0 & -2,5 & 0,3  & 0,8                            \\
S.A. & F & 10,1 &  		& -2,8 & -0,8 & 12,0 & 0,26 & 5,3 & -0,6 & 2,2  & 0,2                            \\
V.D. & M & 13,9 &  		& -1,5 & -1,4 & 11,7 & 0,29 & 1,3 & -0,5 & 1,0  & 0,9                            \\
Z.G. & M & 15,0 &  		&  -2  & -1,7 &   --   & 0,26 & 1,5 & -1,3 & 0,7  & 0,4                            \\
Z.M. & M & 8,5  &  		& -2,2 & -1,0 & 13,0 & 0,34 & 4,9 & -1,7 & 0,5  & -0,7                           \\
Z.L. & M & 9,9  &  		& -2,8 & -2,1 & 11,9 & 0,28 & 5,5 & -2,9 & -0,1 & -0,8                           \\

\bottomrule
\end{tabular}
\end{sidewaystable}