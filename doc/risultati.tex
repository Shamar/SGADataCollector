\chapter{Risultati}

La tabella 4.1 riassume la situazione auxologica dei pazienti all'inizio della terapia ed il loro potenziale genetico. La statura è espressa in centimetri e in Standard Deviation Score; il Mid-Parental Height (MPH) è stato calcolato secondo la formula illustrata precedentemente nella sezione Materiali e metodi. Non disponiamo della statura dei genitori delle pazienti B.M. e P.C. perché sono state adottate. Tutti i pazienti arruolati presentavano una statura inferiore alla norma, con deficit staturale medio di -2,8 SDS e range da -4,8 a -1,5 SDS. Le stature dei genitori erano anch'esse piuttosto basse: la statura media dei padri era di 164,9 cm con un inetrvallo da 144,5 a 180,0 cm; la statura media della madri era di 151,5 cm e range da 137,2 a 177,1; il Mid-Parental Height (MPH) medio era ai limiti inferiori della norma, pari a -1,6 SDS, con un range da -3,9 a 1,2 SDS.

\begin{table}[!h]
\begin{center}
\begin{tabular}{lccccccc}
\toprule
 & \multirow{2}{*}{Sesso} & 
   \multirow{2}{*}{Padre} & 
   \multirow{2}{*}{Madre} & 
   \multicolumn{2}{c}{MPH} &
   \multicolumn{2}{c}{Altezza} \\
 \cmidrule(r){5-6} \cmidrule(r){7-8}
 & & \emph{cm} & \emph{cm} & \emph{cm} & \emph{SDS}  &		\emph{cm}	& \emph{SDS}             	 \\

\midrule
A.A.	& F & 170,6 & 160,4 & 159,0 & -0,5 	 		& 123,3 & -2,6  \\
B.A.	& F & 170,1 & 156,8 & 156,9 & -0,9   			& 103,7 & -2,7   \\
B.M.	& F & --    & --    & --    & --     			&  97,6 & -3,7  \\
C.D.	& M & 157,7 & 147,8 & 159,2 & -2,3 	        	& 116,1 & -3,4  \\
C.B.	& F & 165,7 & 161,2 & 156,9 & -0,9 	 	        & 125,9 & -2,5  \\
D.V.	& F & 159,2 & 149,8 & 148,0 & -2,4   			& 109,5 & -2,5   \\
D.A.	& M & 157,0 & 143,0 & 156,5 & -2,7 	  		& 120,2 & -2,7  \\
D.S.	& F & 154,8 & 155,9 & 148,8 & -2,3   			& 122,8 & -3,6  \\
E.A.	& M & 144,5 & 139,2 & 148,3 & -3,9 	  		& 113,7 & -4,6  \\
G.A.	& F & 168,2 & 137,2 & 146,2 & -2,7 	  		&  88,1 & -3,2  \\
L.A.	& M & 160,4 & 147,1 & 160,2 & -2,1   			& 128,2 & -3    \\
L.R.	& M & 163,3 & 145,6 & 160,9 & -2,0 	  		& 117,7 & -2,9  \\
L.L.	& M & 164,0 & 143,0 & 160,0 & -2,2 	  		& 136,8 & -2     \\
M.V.	& F & 165,5 & 164,2 & 158,3 & -0,6   			& 135,5 & -2,7  \\
M.E.	& F & 163,2 & 142,6 & 146,4 & -2,7 	  		& 122,6 & -4,8  \\
P.G.	& F & 160,6 & 159,4 & 153,5 & -1,5   			& 121,7 & -2,9  \\
P.D.	& M & 174,2 & 151,8 & 169,5 & -0,8 	  		& 121,5 & -2,3  \\
P.C.	& F & --    & --    & --    & --   	  		& 105,8 & -2,8   \\
P.S.	& F & 180,0 & 171,0 & 169,0 & 1,2  	  		& 132,5 & -3    \\
R.M.	& M & 174,5 & 177,1 & 182,3 & 1,1    			& 112,2 & -3,1  \\
S.M.	& F & 172,8 & 143,1 & 151,4 & -1,8 	        	& 122,0 & -2,2  \\
S.F.	& F & 158,0 & 139,8 & 142,4 & -3,4 	  		& 111,6 & -2,8  \\
S.A.	& F & 174,7 & 153,5 & 157,6 & -0,8 	  		& 120,8 & -2,8  \\
T.G.    & M & --    & --    & 171,5 & -0,5    			& 136,9 & -3,0  \\
V.D.	& M & 171,2 & 146,3 & 165,2 & -1,4 	  		& 149,8 & -1,5  \\
Z.G.	& M & 165,5 & 150,0 & 164,2 & -1,7 	  		& 156,0 & -2     \\
Z.M.	& M & 166,0 & 156,8 & 167,9 & -1,0 	  		& 116,9 & -2,2  \\
Z.L.	& M & 161,4 & 146,1 & 160,2 & -2,1 	  		& 120,1 & -2,8  \\
\bottomrule
\end{tabular}
\end{center}
\caption{Potenziale genetico e situazione auxologica iniziale.}
\label{tab:SituazioneIniziale}
\end{table}

L'età media a cui i nostri pazienti hanno cominciato la terapia è stata di dieci anni, con un ampia variabilità: la paziente G.A. ha iniziato all'età di quattro anni con un grave deficit staturale; i pazienti V.D. e Z.G. hanno iniziato la terapia ad un'età tardiva (13,9 e 15 anni rispettivamente) a pubertà già iniziata. Questi dati non si discostano da quelli delle casistiche dell'Europa settentrionale: in uno studio svedese sui bambini SGA in terapia con rGH pubblicato nel 2005 l'età media all'inizio del trattamento era di 10,7 anni\cite{dahlgren2005final}.
In media la terapia è durata complessivamente cinque anni, con un range estremamente variabile da 1,3 a 8,9 anni. La durata media del trattamento nel periodo precedente lo sviluppo puberale è stata di soli 1,9 anni, anche in questo caso con un grande margine di variabilità: alcuni pazienti hanno potuto beneficiare di un lungo periodo di terapia nell'età infantile (circa sette anni), altri hanno iniziato la terapia a pubertà avanzata. In media ai pazienti è stata somministrata una dose pari a 0,27 mg/kg/sett, con un'ampia variabilità: la paziente P.S. ha assunto una dose media di 0,21 mg/kg/sett, mentre il paziente P.D. ha beneficiato di una dose piuttosto elevata, pari a 0,36 mg/kg/sett. 
  
\begin{table}[!h]
\begin{center}
%\addtolength{\tabcolsep}{-1pt}
%\renewcommand{\arraystretch}{1.1}
\begin{tabular}{lccrcccl}
\toprule
 &    \multirow{2}{*}{Sesso} &
 \multirow{2}{*}{Deficit}    & \multirow{2}{*}{Età inizio} & \multirow{2}{*}{Dose media} & \multicolumn{2}{c}{Durata Terapia}\\
\cmidrule(r){6-7}
 &       &   &    &  	& Prepuberale & Totale \\
 &     &	&			& \emph{mg/kg/sett}	& \emph{aa} & \emph{aa} \\
\midrule                                  	
A.A.	& F &  		 & 10,6  	&  0,28     	& 0,0                 & 3,6  \\
B.A.	& F & \checkmark & 6,6   	&  0,26     	& 6,1                 & 8,1  \\
B.M.	& F & \checkmark & 6,4   	&  0,30     	& 6,9                 & 7,6  \\
C.D.	& M &  	  	 & 9,7   	&  0,25     	& 3,0                 & 7,1  \\
C.B.	& F &  		 & 10,7  	&  0,25     	& 0,0                 & 3,5  \\
D.V.	& F & \checkmark & 7,5   	&  0,27     	& 3,6                 & 6,6  \\
D.A.	& M &  		 & 9,9   	&  0,28     	& 0,5                 & 5,8  \\
D.S.	& F & \checkmark & 11,5  	&  0,31     	& 0,5                 & 4,0  \\
E.A.	& M &  		 & 10,8  	&  0,30     	& 2,4                 & 6,4  \\
G.A.	& F &  		 & 4,0   	&  0,26     	& 5,5                 & 8,6  \\
L.A.	& M & \checkmark & 12,0  	&  0,26     	& 1,5                 & 3,5  \\
L.R.	& M &  		 & 9,6   	&  0,23     	& 1,5                 & 5,5  \\
L.L.	& M &  		 & 12,7  	&  0,26     	& 1,5                 & 4,5  \\
M.V.	& F & \checkmark & 12,3  	&  0,26     	& 0,5                 & 3,6  \\
M.E.	& F &  		 & 11,6  	&  0,33     	& 0,5                 & 4,0  \\
P.G.	& F & \checkmark & 10,3  	&  0,26     	& 0,0                 & 4,6  \\
P.D.	& M &  		 & 9,6   	&  0,36     	& 1,9                 & 5,9  \\
P.C.	& F &  		 & 7,5   	&  0,31     	& 2,4                 & 6,2  \\
P.S.	& F &  		 & 12,0  	&  0,21     	& 1,0                 & 3,0  \\
R.M.	& M & \checkmark & 6,9   	&  0,29     	& 3,0                 & 8,9  \\
S.M.	& F &  	  	 & 9,7   	&  0,27     	& 0,5                 & 4,7  \\
S.F.	& F &  		 & 8,3   	&  0,26     	& 1,5                 & 6,0  \\
S.A.	& F &  		 & 10,1  	&  0,26     	& 2,0                 & 5,3  \\
T.G.    & M &            & 13,5         &  0,22         & 1,5                 & 2,4  \\
V.D.	& M &  		 & 13,9  	&  0,29     	& 0,0                 & 1,3  \\
Z.G.	& M &  		 & 15,0  	&  0,26     	& 0,0                 & 1,5  \\
Z.M.	& M &  		 & 8,5   	&  0,34     	& 4,5                 & 4,9  \\
Z.L.	& M &  		 & 9,9   	&  0,28     	& 2,0                 & 5,5  \\
\bottomrule
\end{tabular}
\end{center}
\caption{Terapia.}
\label{tab:Terapia}
\end{table}

La pubertà nei maschi è iniziata ad un'età media di 12,4 anni, con una variabilità piuttosto ampia: il paziente R.M. ha iniziato lo sviluppo a circa dieci anni, mentre il paziente T.G. all'età di quindici anni. Nelle femmine la pubertà è iniziata in media a 11,3 anni, con un range da 9,5 anni a 13,3 anni. Nessuno dei pazienti in esame ha presentato una pubertà precoce vera. Sette pazienti (cinque femmine e due maschi) hanno effettuato la terapia antipuberale con un analogo sintetico del LHRH perché presentavano un modesto anticipo puberale e soprattutto perché all'inizio dello sviluppo avevano una statura molto bassa.
 
\begin{table}[!h]
\begin{center}
%\addtolength{\tabcolsep}{12pt}
%\renewcommand{\arraystretch}{1.1}
\begin{tabular}{lccccc}
\toprule
 & Sesso 	& \multicolumn{1}{c}{Età inizio}	& Altezza	& Terapia frenante & Scatto puberale \\
 & &  \emph{aa} 	& \emph{cm}	& & \emph{cm}		\\
\midrule
A.A.	& F & 10,2 		& 121,8  		& \checkmark & 24,2 \\
B.A.	& F & 12,7 		& 137,9   		&            & 11,1 \\
B.M.	& F & 13,3 		& 141,5   		&            & 19,3 \\
C.D.	& M & 12,8 		& 136,2   		&            & 26,7 \\
C.B.	& F & 10,5 		& 125,0   		& \checkmark & 30,6 \\
D.V.	& F & 11,1 		& 131,0   		&            & 15,0 \\
D.A.	& M & 10,4 		& 123,6   		& \checkmark & 35,4 \\
D.S.	& F & 12,0 		& 126,3   		&            & 18,4 \\
E.A.	& M & 13,3		& 125,7   		&            & 21,6 \\
G.A.	& F &  9,5  		& 122,5   		&            & 20,5 \\
L.A.	& M & 13,5 		& 134,3   		&            & 25,9 \\
L.R.	& M & 11,1 		& 127,5   		&            & 33,9 \\
L.L.	& M & 14,2 		& 146,6   		&            & 20,7 \\
M.V.	& F & 12,8 		& 139,3   		&            & 18,2 \\
M.E.	& F & 12,1 		& 126,8   		&            & 20,0 \\
P.G.	& F & 10,0 		& 121,1   		& \checkmark & 26,4 \\
P.D.	& M & 11,5 		& 132,9   		&            & 24,5 \\
P.C.	& F &  9,9  		& 123,8   		& \checkmark & 20,4 \\
P.S.	& F & 13,0 		& 140,0   		&            & 18,0 \\
R.M.	& M &  9,9  		& 135,6   		& \checkmark & 35,8 \\
S.M.	& F & 10,3 		& 126,8   		& \checkmark & 21,2 \\
S.F.	& F &  9,8  		& 124,1   		&            & 23,1 \\
S.A.	& F & 12,0 		& 137,8   		&            & 20,8 \\
T.G.    & M & 15,0      	& 146,5         	&            & 18,0 \\ 
V.D.	& M & 11,7 		& 138,4   		&            & 33,1 \\
Z.G.	& M &  --    		& --   			&            & --   \\
Z.M.	& M & 13,0 		& 139,5   		&            & 23,5 \\
Z.L.	& M & 11,9 		& 132,6   		&            & 22,4 \\
\bottomrule
\end{tabular}
\end{center}
\caption{Pubertà.}
\label{tab:Puberta}
\end{table}

In media i pazienti hanno effettuato uno scatto puberale soddisfacente: mediamente i maschi hanno acquisito 26,8 cm con lo sviluppo, con un range da 18,0 a 35,8 cm; mentre le femmine sono cresciute in media 20,5 cm, con un intervallo da 11,1 a 30,5 cm. Per confronto, nella poplazione generale lo scatto di crescita puberale medio è pari a 23 cm per i maschi e 20 cm per le femmine.  \cite{tanner1990foetus}

\begin{table}[!h]
\begin{center}
%\addtolength{\tabcolsep}{12pt}
%\renewcommand{\arraystretch}{1.1}
\begin{tabular}{lccccc}
\toprule
 & Sesso & \multicolumn{2}{c}{Altezza finale} 	& SDS dal MPH	& \Delta SDS \\
\cmidrule(r){3-4}
  &  & \emph{cm} 	& \emph{SDS}  	   	&		& 			 	\\
\midrule
A.A.	& F & 146,0 & -2,7  & -2,2  & -0,1    \\
B.A.	& F & 149,0 & -2,2  & -1,4  & 0,5     \\
B.M.	& F & 160,8 & -0,2  & --    & 3,5     \\
C.D.	& M & 162,9 & -1,8  & 0,5   & 1,6     \\
C.B.	& F & 155,6 & -1,1  & -0,2  & 1,4     \\
D.V.	& F & 146,0 & -2,7  & -0,3  & -0,2    \\
D.A.	& M & 159,0 & -2,5  & 0,2   & 0,2      \\
D.S.	& F & 144,7 & -3,0  & -0,7  & 0,6     \\
E.A.	& M & 147,3 & -4,0  & -0,1  & 0,6     \\
G.A.	& F & 143,0 & -3,5  & -0,8  & -0,3        \\
L.A.	& M & 160,2 & -2,1  & 0,0   & 0,9     \\
L.R.	& M & 161,4 & -2,0  & 0,1   & 0,9     \\
L.L.	& M & 167,3 & -1,1  & 1,1   & 0,9     \\
M.V.	& F & 157,5 & -0,8  & -0,1  & 1,9     \\
M.E.	& F & 146,8 & -2,6  & 0,1   & 2,2     \\
P.G.	& F & 147,5 & -2,5  & -1,0  & 0,4     \\
P.D.	& M & 157,4 & -2,5  & -1,8  & -0,2    \\
P.C.	& F & 144,2 & -3,0  & --  & -0,2   \\
P.S.	& F & 158,0 & -0,7  & -1,9  & 2,3     \\
R.M.	& M & 171,4 & -0,5  & -1,6  & 2,6    \\
S.M.	& F & 148,0 & -2,4  & -0,6  & -0,2    \\
S.F.	& F & 147,2 & -2,5  & 0,8   & 0,3     \\
S.A.	& F & 158,6 & -0,6  & 0,2   & 2,2     \\
T.G.    & M & 164,5 & -1,6  & -1,1  & 1,4    \\
V.D.	& M & 171,5 & -0,5  & 0,9   & 1,0     \\
Z.G.	& M & 166,3 & -1,3  & 0,4   & 0,7   \\
Z.M.	& M & 163,0 & -1,7  & -0,7  & 0,5     \\
Z.L.	& M & 155,0 & -2,9  & -0,8  & -0,1    \\
\bottomrule
\end{tabular}
\end{center}
\caption{Esito.}
\label{tab:Esito}
\end{table}

Per quanto concerne la statura finale, in media è stata raggiunta una statura ai limiti inferiori della curva di normalutà per la poplazione generale, anche qui con un ampio margine di variabilità: il paziente E.A. non ha normalizzato la sua statura definitiva, che è risultata inferiore di 4 SDS rispetto alla media della popolazione generale; sei pazienti hanno raggiunto una statura finale che si discosta meno di una SDS dalla statura media della popolazione generale. 

\begin{table}[!h]
\begin{center}
%\addtolength{\tabcolsep}{-4pt}
\begin{tabular}{l c c c clrclr}
\toprule
                                & & & & \multicolumn{6}{c}{Media} \\ 
\midrule                        
Statura padre\emph{(cm)}        & & & & 164,9 & & (144,5 &\div & 180,0)& \\
Statura madre\emph{(cm)}        & & & & 151,5 & & (137,2 &\div & 177,1)& \\
MPH      \emph{(SDS)}           & & & & -1,6 & &(-3,9 &\div &  1,2)&  \\ 
Statura inizio terapia \emph{(SDS)}         & & & & -2,8 & &(-4,8 &\div & -1,5)& \\ 
Età inizio terapia  \emph{(aa)}            & & & & 10,0 & &(4,0  &\div & 15,0)& \\ 
Dose \emph{(mg/kg/sett)}        & & & & 0,27 & &(0,21 &\div & 0,36)& \\ 
Anni prepuberali trattati  \emph{(aa)}     & & & & 1,9  & &(0    &\div &  6,9)&  \\ 
Durata della terapia \emph{(aa)}& & & & 5    & &(1,3  &\div &  8,9)&  \\ 
Final Height \emph{(SDS)}       & & & & -2,0 & &(-4,0 &\div & -0,2)& \\ 
SDS dal MPH                     & & & & -0,4 & &(-2,2 &\div &  1,1)&  \\ 
\Delta SDS                      & & & & 0,9  & &(-0,3 &\div &  3,5)&  \\ 
\bottomrule
\end{tabular} 
\end{center}
\caption{Medie.}
\label{tab:Medie}
\end{table}

\begin{table}[!h]
\begin{center}
\addtolength{\tabcolsep}{-2pt}
\begin{tabular}{l c c c clrclr c clrclr}
\toprule
 & & & &\multicolumn{6}{c}{Maschi}& & \multicolumn{6}{c}{Femmine}\\
\midrule
Età Inizio \emph{(aa)}             & & & & 12,4  &  & (9,9    &\div &  15,0) &   & & 11,3  &  &(9,5    &\div &  13,3) &  \\ 
Altezza inizio \emph{(cm)}         & & & & 135,0 &  & (123,6  &\div & 146,6) &   & & 129,8 &  &(121,1  &\div & 141,5) &  \\ 
Scatto puberale \emph{(cm)}        & & & & 26,8  &  & (18,0   &\div &  35,8) &   & & 20,5  &  &(11,1   &\div &  30,6) &  \\ 
\bottomrule
\end{tabular} 
\end{center}
\caption{Medie relative alla pubertà (per sesso).}
\label{tab:MediePerSesso}
\end{table}

Se confrontiamo la statura definitiva dei pazienti con il rispettivo MPH, cioè con la statura più probabile per il potenziale genetico, possiamo osservare che in media i pazienti hanno pressochè raggiunto il centro della fascia-bersaglio parentale (MPH). Solamente una paziente ha raggiunto una statura definitiva al limite inferiore del proprio potenziale genetico (-2,2 SDS dal MPH), mentre nove pazienti hanno superato il loro MPH. 
In media i pazienti hanno guadagnato circa una SDS rispetto alla stutura all'inizio del trattamento, con un'ampia variabilità: la paziente G.A. ha riportato un lieve peggioramento, mentre la paziente B.M. ha presenato un guadagno importante pari a 3,5 SDS.
  


\begin{table}[!h]
\begin{center}
%\addtolength{\tabcolsep}{-4pt}
\begin{tabular}{l c c c c}
\toprule
\multirow{2}{*}{Modello}& \multirow{2}{*}{Sig.} & \multicolumn{2}{c}{Intervallo di confidenza 95,0\%} \\
			& 	& Limite inferiore & Limite superiore \\
\midrule
MPH			& 0,008	& 0,021	  & 0,117 \\
Sesso			& 0,022	& 0,194	  & 2,205 \\
Deficit GH		& 0,737	& -0,925  & 0,668 \\
Età inizio		& 0,087	& -0,067  & 0,888 \\
Dose media		& 0,596	& -12,386 & 7,368 \\
dt Prepuberale		& 0,730	& -0,560  & 0,401 \\
durataTotale		& 0,058	& -0,017  & 0,921 \\
Terapia Frenante	& 0,388	& -1,665  & 0,684 \\
\bottomrule
\end{tabular} 
\end{center}
\caption{Coefficienti di regressione lineare multivariata rispetto alla variabile dipendente definita dal guadagno in SDS.}
\label{tab:Statistiche}
\end{table}

