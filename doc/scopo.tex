 \chapter{Scopo della tesi}

\section{Premesse: la terapia con GH nei bambini SGA}


Il GH ("growth hormone", ormone della crescita) agisce a livello di diversi organi e apparati.
L'azione principale nell'infanzia e nell'adolescenza si esplica attraverso l'accrescimento
longitudinale delle ossa, stimolando la proliferazione e la differenziazione dei condrociti
a livello delle cartilagini di accrescimento epifisarie con successiva calcificazione
e incorporazione nell'osso metafisario. Esso stimola inoltre la sintesi del collagene 
di tipo I e la proliferazione osteoblastica.\cite{sga}


Nonostante non fosse ancora chiara l'eziologia della bassa statura nei soggetti SGA,
negli anni '70 si è iniziato l'approccio terapeutico con l'ormone della crescita.

In particolare nel 1969 vennero trattati per la prima volta con ormone della crescita  due bambini nati piccoli per l'età gestazionale affetti dalla sindrome di Silver Russell. Durante il trattamento i due pazienti mostrarono un significativo aumento della velocità di crescita staturale e si avvicinarono al terzo centile delle carte staturali di distanza. Purtroppo non furono riportati i dati sulla statura finale\cite{tanner1969low}.

Nel 1974 furono trattati con GH otto bambini con ritardo di crescita intrauterino. Anche in questo studio venne riportato un incremento della velocità di crescita degno di nota durante la terapia\cite{lee1974growth}.


Visti gli incoraggianti risultati a breve termine e notando che altri pazienti senza deficit di GH traevano beneficio dalla terapia con ormone somatotropo (ad esempio le bambine con sindrome di Turner\cite{betts1999decade}),
si proseguirono gli studi sui bambini SGA in trattamento con GH.


Nel 2009 è stata effettuata una meta-analisi di ventinove studi prevalentemente multicentrici su bambini SGA  con lo scopo di valutare l'impatto del trattamento con rGH sulla statura definitiva. Sono stati inclusi anche i bambini nati piccoli per l'età gestazionale  affetti dalla sindrome di Silver Russell. Di questi ventinove studi solo nove sono stati considerati validi in termini di durata della terapia, continuità della stessa ed obiettivo (impatto della terapia sulla statura finale)\cite{maiorana2009impact}.

Dunque finora pochissimi studi hanno valutato l'effettivo impatto della terapia continua e di lunga durata con rGH sulla statura definitiva dei bambini nati SGA . 


\section{Obiettivi dello studio}

Lo scopo della mia tesi è presentare i risultati della terapia continua e a lungo termine  con GH biosintetico sulla statura finale dei soggetti nati piccoli per l'età gestazionale.

In particolare ho inteso valutare se la terapia prolungata con rGH porta alla normalizzazione della statura adulta (vale a dire raggiungimento di una statura compresa fre le +2 e le -2 SDS) e soprattutto se tale terapia permette ai pazienti di raggiungere una statura finale compresa nella fascia-bersaglio parentale. 

L'interesse dello studio è anche suscitato dalle recenti disposizioni legislative ( si veda in Appendice la Nota 39 AIFA del 18 Novembre 2010), le quali permettono di trattare i bambini SGA che non abbiano normalizzato la statura all'età di quattro anni. 

L'utilità della terapia con rGH è stata da me valutata mediante la determinazione dei seguenti indici:

\begin{itemize}
\item la statura finale (Final Height, FH) espressa in Standard Deviation Score;
\item la statura finale  corretta per il centro della fascia-bersaglio parentale (Mid-Parental Height, MPH) espressa in SDS;
\item le SDS guadagnate, confrontando la statura finale in SDS con la statura all'inzio del trattamento espressa anch'essa in SDS.
\end{itemize}

L'espressione della statura in SDS permette di annullare le differenze dovute al sesso e all'età.

