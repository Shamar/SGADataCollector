 \chapter{Scopo della ricerca}

\section{Premesse}


Il GH ("growth hormone", ormone della crescita) agisce a livello di diversi organi e apparati.
L'azione principale nell'infanzia e nell'adolescenza si esplica attraverso l'accrescimento
longitudinale delle ossa, stimolando la proliferazione e la differenziazione dei condrociti
a livello delle cartilagini di accrescimento epifisarie con successiva calcificazione
e incorporazione nell'osso metafisario. Esso stimola inoltre la sintesi del collagene 
di tipo I e la proliferazione osteoblastica.\cite{sga}


Nonostante non fosse ancora chiara l'eziologia della bassa statura nei soggetti SGA
negli anni '70 si è iniziato l'approccio terapeutico con l'ormone della crescita biosintetico.


Inizialmente i risultati non furono brillanti, probabilmente per il trattamento non 
ottimale in termini di dosaggio e frequenza di somministrazione. Nonostante ciò i 
bambini SGA trattati con l'ormone somatotropo per due, tre anni crescevano meglio
rispetto a quelli non trattati. Sulla base di tali osservazioni si è proseguito con
il trattamento, modificandone il dosaggio e la frequenza di somministrazione.
 

Gli studi condotti negli ultimi 30 anni hanno permesso di verificare l'efficacia,
l'utilit{\`a}, la sicurezza e soprattutto l'adeguatezza della dose dell'ormone somatotropo
nei bambini nati piccoli per l'et{\`a} gestazionale.


In particolare:

\begin{description}
\item[Anita Hokken-Koelega et al. (2003)] \hfill \\
si tratta di uno studio multicentrico che ha preso in considerazione 165 bambini SGA prepuberi.75 di questi sono stati assegnati casualmente alla valutazione della risposta al trattamento con ormone della crescita biosintetico alla dose di 0.033 mg/kg/die (gruppo A,n=41) e alla dose di 0.067 mg/kg/die (gruppo B, n=38). I restanti 90 bambini sono stati casualmente sottoposti ad una dose di GH pari a 0.033 mg/kg/die (n=60) o destinati al gruppo di controllo (n=30). Scopo dello studio era valutare gli effetti della terapia con GH su composizione corporea, metabolismo dei carboidrati e altezza finale nei soggetti SGA.Lo studio è durato 6 anni, al termine dei quali è emerso che la terapia con GH per lungo periodo e senza interruzioni porta alla normalizzazione della statura nell'infanzia e ad una normale statura finale nella maggior parte dei bambini, indipendentemente dalla dose. Solo bambini molto bassi o di età maggiore necessitano di dosaggi alti. Inoltre si è dimostrato che la terapia con GH non ha effetti avversi sui livelli di glucosio e sui lipidi sierici ed ha un effetto positivo sulla pressione arteriosa. Tuttavia il GH induce un innalzamento dei livelli  di insulina, segno di insulino-resistenza; i livelli di insulina tornano nell'intervallo di normalità con la cessazione della terapia. Infine si è visto che la terapia con GH migliora significativamente la mineralizzazione ossea e la massa magra dei bambini nati SGA. 

\item[Yvonne Van Pareren et al. (2003)] \hfill \\
 questo studio multicentrico si è posto come obiettivo la valutazione dell'effetto di un trattamento continuo ed a lungo termine con GH sulla statura finale di 54 bambini bassi nati SGA.I bambini,alcuni dei quali risultati parzialmente  deficitari di GH, sono stati suddivisi casualmente in due gruppi: uno trattato con GH alla dose di 3 UI (0.033 mg/kg/die); l'altro trattato con una dose maggiore (6 UI = 0.067 mg/kg/die). Il trattamento è durato circa 2 anni. Come controllo si è scelto un gruppo di bambini nati SGA non deficitari di GH i cui pediatri si sono opposti al trattamento ormonale. Si è osservato che, indipendentemente dalla produzione di GH prima del trattamento e a prescindere dalla dose assunta nel corso del trattamento, la maggior parte dei bambini ha raggiunto una statura finale compresa nell'intervallo di normalità e la quasi totalità 
 ha raggiunto una statura entro i limiti del bersaglio parentale. Solo bambini SGA estremamente bassi o con centro target al di sotto dell' intervallo di normalità potrebbero necessitare di alte dosi di ormone della crescita. Gli autori concludono dicendo che sono necessari ulteriori studi per ottimizzare il trattamento con GH dei bambini SGA al fine di individuare la migliore opzione terapeutica per ogni bambino. 

\item[Myriam Rosilio et al. (2005)] \hfill \\
 si tratta di uno studio multicentrico condotto su 35 bambini prepuberi nati SGA. I bambini hanno ricevuto quotidianamente per due anni una dose di ormone delle crescita biosintetico piuttosto alta (0.067 mg/kg/die). Alcuni di questi bambini hanno riassunto l'ormone dopo una pausa di 2 anni. Non sono emerse alterazioni riguardo al metabolismo del glucosio durante il trattamento e nel successivo follow-up. è stata riportata l'altezza finale di 20 dei 35 soggetti inclusi nello studio. I modesti risultai ottenuti in termini di SDS guadagnate sono probabilmente attribuibili alla breve durata del trattamento ed al suo schema discontinuo. Gli autori affermano che ulteriori studi sono necessari per chiarire la risposta al trattamento ed individualizzarlo.

\item[Arianna Maiorana e Stefano Cianfarani (2009)] \hfill \\
 è una meta-analisi di 29 studi su bambini SGA. Di questi solamente quattro vengono considerati validi in termini di durata della terapia (superiore ai due anni), continuità della stessa ed obiettivo ( impatto della terapia sulla statura finale).  Da questi quattro studi  emerge che la terapia con GH sembra essere efficace nel ridurre almeno in parte il deficit staturale che i soggetti SGA presentano in età adulta, senza differenze statisticamente significative fra i due regimi di dosi terapeutiche (0.033 vs 0.067 mg/kg/die). Gli autori concludono però che la risposta alla terapia è altamente variabile ed occorrono ulteriori studi per identificare coloro che rispondono alla terapia con un significativo miglioramento della statura finale.  

\end{description}

\nocite{hokken2003final,van2003adult,rosilio2005adult,maiorana2009impact}

Da quanto esposto sopra emerge che :
\begin{itemize}
\item la terapia con GH non induce diabete nei soggetti nati SGA, i quali rappresentano una poplazione a rischio di complicanze glicometaboliche.
\item attualmente gli studi valutanti l'impatto di una terapia continua e di lunga durata 
con ormone somatotropo biosintetico sulla statura finale di bambini nati piccoli per l'età gestazionale sono pochi.
\end{itemize}

Di fatto però i soggetti nati SGA costituiscono una componente importante degli adulti di bassa statura e rappresetano una popolazione destinata ad aumentare numericamente se si considerano i fattori di rischio per la nascita di bambini SGA (vedi introduzione).
 
 


%In Italia è possibile la prescrizione a carico del Servizio Sanitario Nazionale,
%secondo le direttive pubblicate sulla Gazzetta Ufficiale del 13 ottobre 2009 che 
%attraverso la Nota 39 ha uniformato sul territtorio nazionale le condizioni 
%richieste dalle Commissioni Regionali preposte per l'autorizzazione al trattamento.(in casistica)



\section{Obbiettivi dello studio}

Lo scopo del presente studio è presentare i risultati della terapia continua e a lungo termine  con GH biosintetico sulla statura finale dei soggetti nati piccoli per l'età gestazionale.

In particolare vengono considerati:

\begin{itemize}
\item l'altezza finale (final height, FH) espressa in Standard Deviation Score;
\item le SDS guadagnate, confrontando l'altezza finale in SDS con l'altezza all'inzio del trattamento espressa anch'essa in SDS;
\item l'altezza finale espressa in SDS e le SDS guadagnate corrette per il centro del bersaglio parentale (midparental height).
\end{itemize}



