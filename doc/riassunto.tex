\documentclass[12pt,a4paper]{article}
\usepackage[utf8x]{inputenc}
\usepackage{ucs}
\usepackage{amsmath}
\usepackage{amsfonts}
\usepackage{amssymb}
\usepackage{setspace}  % package per l'interlinea
\onehalfspacing		% interlinea 1,5 con setspace
\author{Sara Bertolotto, Matricola 239830}
\title{Impatto della terapia con ormone somatotropo sulla statura finale dei bambini nati piccoli per l'età gestazionale}
\date{Anno Accademico 2010 -- 2011}
\begin{document}
\maketitle
In base ai dati antropometrici rilevati alla nascita, un neonato è definito piccolo per l'età gestazionale quando presenta peso e/o lunghezza inferiori al 3° percentile o alle - 2 SDS rispetto alle curve di normalità per sesso relative alla popolazione di appartenenza. Circa il 5\% dei nati in tutto il mondo sono piccoli per l'età gestazionale. 

I nati SGA costituiscono un gruppo eterogeneo per eziologia. Nei Paesi poveri le principali cause di iposviluppo fetale sono: le infezioni, il fumo, l'inadeguata nutrizione materna, la giovane età della gravida, l'eccessivo carico di lavoro della madre. Nei Paesi sviluppati le più frequenti ragioni comprendono: la gemellarità (favorita dalle pratiche di fecondazione assistita), il fumo, la dieta eccessivamente rigida della gestante, l'avanzata atà materna e le disfunzioni placentari.

Dal punto di vista clinico gli SGA vengono suddivisi in asimmetrici e simmetrici. Si parla di SGA asimmetrico quando è inferiore alla norma solo il peso; si definisce SGA simmetrico il neonato che presenta compromissione di tutte le variabili antropometriche: peso, lunghezza e circonferenza cranica.

Osservazioni epidemiologiche hanno evidenziato che i soggetti SGA presentano un aumentato rischio di alterazioni metaboliche e cardiovascolari in età adulta. Inoltre il 10-15\% di essi è destinato a non raggiungere una statura normale.

Le cause di bassa statura nei soggetti SGA comprendono: mancata o incompleta crescita di recupero; pubertà sovente anticipata e/o a rapida progressione; scatto puberale di crescita ridotto; maturazione scheletrica inizialmente ritardata seguita da rapida accelerazione. Infine una percentuale di SGA presenta insufficienti livelli di ormone somatotropo.

I soggetti nati SGA costituiscono una componente importante degli adulti di bassa statura (25\%).

Negli anni '70 si è iniziato a trattare questi bambini con ormone della crescita. Tuttavia fino ad oggi solo pochi studi hanno riportato l'effettivo impatto della terapia sulla statura definitiva dei bambini SGA.

La scopo della mia tesi è presentare i risultati del trattamento continuo e a lungo termine con GH  biosintetico sulla statura finale dei bambini nati SGA.

La mia casistica è composta da 28 soggetti nati SGA, non sindromoci, affetti da bassa statura e trattati con rGH presso la Struttura Semplice di Auxologia del Dipartimento di Discipline Pediatriche e dell'Adolescenza dell'Università di Torino.

I soggetti in esame hanno effettuato in media 5 anni di terapia, guadagnando circa 1 SDS rispetto alla statura di inizio terapia. Si tratta di un buon risultato, che ha permesso loro di raggiungere una statura definitiva ai limiti inferiori della norma, ben compatibile con il loro potenziale genetico. Infatti la maggior parte dei pazienti presentava anche bassa statura familiare. 

Un risultato ancora migliore si sarebbe ottenuto iniziando la terapia in età infantile, periodo in cui l'individuo realizza buona parte della sua statura definitiva, ma la maggioranza dei nostri pazienti ha intrapreso il trattamento con rGH poco prima dell'ingresso in pubertà o a sviluppo già iniziato.

Dunque il nostro risultato avvalora la attuali disposizioni legislative, secondo le quali i bambini SGA che non abbiano normalizzato la propria statura a quattro anni di età possono accedere alla terapia con ormone della crescita biosintetico.  



 
\end{document}