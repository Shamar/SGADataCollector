\chapter{Casistica, materiali e metodi}

\section{Casistica}

Sono stati selezionati 28 pazienti (12 maschi e 16 femmine) nati SGA, affetti da bassa statura e trattati con rGH presso l'Istituto di Discipline Pediatriche dell'Università  di Torino.

Al fine di ottenere una casistica il più possibile omogenea, si sono adottati i seguenti criteri di inclusione:

\begin{itemize}
\item peso e/o lunghezza alla nascita inferiori al terzo centile secondo gli standarad antropometrici neonatali prodotti dalla task-force della Società Italana di Neonatologia e basati su una popolazione italiana nord-orientale (Figura~\ref{fig:StandardNeonataliNordOccidentali});
\begin{figure}[!h]
  \begin{center}
      \includegraphics{grafici/centili/centili} %\\
  \end{center}
  \caption{Standard antropometrici neonatali relativi a peso ed altezza nell'Italia nord-orientale}%citare l'articolo
  \label{fig:StandardNeonataliNordOccidentali}
\end{figure}
\item assenza di anomalie genetiche (ad esempio Sindrome di Silver Russell, Sindrome di Noonan),sindromi genetiche conosciute (ad esempio Sindrome di Turner), malattie croniche che limitano la crescita (ad esempio insufficienza renale cronica, malattia celiaca), dismorfismi importanti, displasie scheletriche di rilievo;
 \item anamnesi negativa per l'assunzione di farmaci che interferiscono con l'accrescimento (ad esempio corticosteroidi assunti ad alte dosi per un lungo periodo di tempo);
 \item assenza di traumi cranici;
 \item anamnesi negativa per malnutrizione e disturbi psicosociali;
 \item misurazione delle stature di entrambi i genitori con statimetro di Harpenden, ad eccezione dei genitori naturali di due soggetti adottati;
\item trattamento con rGH per almeno 1,3 anni consecutivi;
\item raggiungimento della statura definitiva, misurata con statimetro di Harpenden. 
\end{itemize}

In particolare tutti pazienti sono nati nella regione Piemonte (eccetto una, nata in Spagna, ma cresciuta in Piemonte) a partire dall'anno 1983; di tutti i pazienti sono note le misure neonatali.

Tutti i pazienti sono stati sottoposti ad esami ematochimici (emocromo, glicemia, creatininemia, dosaggio delle transaminasi epatiche plasmatiche, dosaggio di IgA totali, anticorpi anti-endomisio e anti-transglutaminasi tissutale, concentrazione ematica di proteina C reattiva) al fine di escludere bassa statura da altre cause. 

Si è valutato il cariotipo dei soggetti di sesso femminile, per escludere la sindrome di Turner (46 X0). 

Nessuno dei pazienti in esame soffriva di anoressia nervosa.


\section{Materiali e metodi}

\subsection{Valutazione auxologica}
Tutti i pazienti sono stati sottoposti a valutazione auxologica prima dell'inizio della terapia, durante e al termine di questa . Infine sono stati valutati a fine crescita. Le misurazioni durante la terapia sono state effettuate a distanza di sei mesi una dall'altra.

\subsubsection*{Età decimale}
L'età è stata espressa in anni e decimi di anno, così da semplificare il calcolo della velocità di crescita. Secondo il calendario decimale l'anno viene diviso in millesimi, così che ogni data corrisponde ad un numero decimale. Ad esempio il 1 maggio 1999 corrisponde a 99,329. La data di nascita viene registrata in modo analogo. L'età ad ogni misurazione corrisponderà alla differenza fra il decimale della data della misurazione e il decimale della data di nascita. %citare foetus into man. riportare la tabella dei decimi di anno?   


\subsubsection*{Statura da eretto}
 La misurazione è stata effettuata con lo statimetro di Harpenden, secondo la tecnica di Cameron% citare libro di Cameron (ho il testo originale)
; la misura,espressa in centimetri e millimetri, è la media delle misure eseguite da due operatori. Sono stati utilizzati gli standard per l'altezza di Tanner.%citare Tanner 1966 pag. 173 auxology) 

  
\subsubsection*{Peso}
 \'E stato rilevato mediante una bilancia tarata che assicurava un'approssimazione di kg 0,1. Tutti i soggetti sono stati pesati con il minimo di indumenti.

\subsubsection*{Velocità di crescita staturale}
Per calcolarla occorre dividere la differenza fra due misurazioni per l'intervallo di tempo fra queste misurazioni. L'intervallo non deve essere nè troppo breve nè molto lungo: in genere la velocità di crescita staturale si calcola su 6 mesi. Sono stati utilizzati gli standard per la velocità di crescita staturale di Tanner%citare Tanner 1966 come sopra

\subsubsection*{Stadi puberali}
Sono un'importante indicatore di maturazione nel periodo puberale. Nei maschi si rilevano lo stadio dello sviluppo pilifero al pube e dello sviluppo dei genitali; nelle femmine lo sviluppo pilfero al pube e lo sviluppo mammario. Generalmente viene adottata la classificazione di Tanner\cite{tanner1990foetus} in cinque stadi; l'inzio della pubertà nella femmina avviene quando lo sviluppo mammario è allo stadio B2 (bottone mammario). 

\subsubsection*{Orchidometria}
Consiste nella valutazione del volume testicolare. A tale scopo è stato utilizzato l'orchidometro di Prader. Nel maschio la pubertà inizia quando il volume testicolare è pari a 4 ml.  


\subsubsection*{Statura bersaglio}
Si tratta dell'altezza che ci si attende nella maggior parte dei figli sani di una coppia con una data statura. Il centro del bersaglio (midparental height) si ottiene nel modo seguente: per i maschi si sommano l'altezza del padre e quella della madre espressa al maschile (cioè si aggiungono 13 centimetri all'altezza materna, perchè tale è la differenza statistica fra maschi e femmine) e si divide il risultato per due; per le femmine si sommano l'altezza della madre e quella del padre espressa al maschile (vale a dire si sottraggono 13 centimetri alla statura paterna per il motivo sopra esposto) e si divide il risultato per due. 

Il centro del bersaglio $\pm$ 8 cm fornisce l'intervallo (fascia-bersaglio) entro cui dovrebbe situarsi la maggior parte dei figli di una coppia con quella data statura.
%citare foetus into man

\subsubsection*{Maturazione scheletrica}
Viene valutata considerando la radiografia di mano e polso sinistri. L'età ossea corrispondente è stata letta da due operatori secondo il sistema TW2 di Tanner e Whitehouse%citare Tanner 1983 (pag. 159 auxology)


\subsubsection*{Statura definitiva}
La statura raggiunta è definitiva quando la velocità di crescita è inferiore ai 2 cm/anno o l'età ossea del soggetto è pari a 15 anni nella femmina, 16 nel maschio.


\clearpage

\subsection{Valutazione ormonale}

Non esiste un \emph{gold standard} per la diagnosi di deficit nella produzione di GH\cite{gh2003update}. La sua secrezione, infatti, è un \emph{continuum} tra normalità e deficit grave. Pertanto i cut-off convenzionalmente usati sono arbitrari e, sebbene i bambini con grave deficit di GH non rispondano ai test di stimolo, non ci sono dubbi che alcuni bambini deficitari di GH ottengano concentrazioni di ormone somatotropo stimolate superiori ai limiti arbitrari applicati.
In passato (nota CUF 39) si parlava di deficit di GH a patogenesi ipofisaria qualora il soggetto non avesse risposto a due test provocativi classici (picco di GH inferiore ai 10 \unit{\micro g}/l dopo stimolo con arginina, glucagone o clonidina) oppure ad un test massimale con GHRH + arginina (picco inferiore a 20 \unit{\micro g}/l); il deficit era definito a patogenesi ipotalamica quando la secrezione somatotropa spontanea media era inferiore ai 3 \unit{\micro g}/l; si parlava di deficit di attività biologica del GH in presenza di bassi livelli di somatomedine (IGF-1) normoresponsivi al test di generazione somatomedinica in pazienti con normale secrezione somatotropa spontanea o stimolata.
L'attuale nota AIFA 39 (18/11/2010) non prende più in considerazione il deficit a patogenesi ipotalamica e quello di attività biologica, ma prevede il trattamento dei bambini nati SGA alle seguenti condizioni: il peso alla nascita deve essere inferiore al terzo centile o comunque al di sotto dei 2500 g; la statura all'inizio della terapia deve trovarsi a -2,5 DS dalla media e la velocità di crescita non deve superare il cinquantesimo centile; il trattamento deve avvenire non prima dei quattro anni di età; sono concessi per ora solo due anni di terapia, in quanto mancano studi che riportino l'impatto della terapia con rGH sulla statura finale dei bambini nati piccoli per l'età gestazionale.

\subsubsection*{L'arginina}
L'arginina è un aminoacido che stimola la secrezione di ormone della crescita verosimilmente inibendo la somatostaina ipotalamica. Viene infusa sottoforma di arginina cloridrato alla dose di 0,5 g/kg (fino ad un massimo di 30 g) in 30 minuti per via endovenosa. I prelievi ematici vengono attualmente effettuati a 0-30-60-90 minuti; precedentemente si eseguivano dopo 30-60-90-120 minuti. L'infusine di arginina non presenta effetti collaterali, esclusa una possibile reazione infiammatoria locale da stravaso.

\subsubsection*{Il glucagone}
Il glucagone stimola il rilascio di GH con meccanismo diretto, analogalmente ai peptidi GH liberatori ed indiretto, mediante induzione di ipoglicemia secondaria. Si somministrano 50 \unit{\micro g}/kg (fino ad un massimo di 1 mg) sottocute o intramuscolo. I prelievi ematici per dosare l'ormone della crescita oggi si effettuano a 0-60-90-120-150 minuti; precedentemente si esguivano a 90-120-150-180 minuti. La somministrazione di glucagone può provocare nausea, vomito; è bene fornire un pasto ricco di carboidrati alla fine del prelievo ematico per ridurre il rischio di ipoglicemia tardiva.

\subsubsection*{La clonidina}
La clonidina è un agonista alfa2adrenergico, che stimola i neuroni produttori di GHRH ed inibisce la secrezione ipotalamica di somatostatina. La si somministra per via orale, sciolta in acqua, alla dose di 0,1-0,5 mg/m2. I prelievi ematici si eseguono dopo 0-30-60-90-120 minuti; in passato si effettuavano dopo 30-60-90-120 minuti. La clonidina può dara effetti collaterali a volte anche gravi: ipotensione marcata, bradiaritmia, sonnolenza.
 
\subsubsection*{Il GHRH + arginina}
Il meccanismo d'azione è duplice: il GHRH induce la liberazione di ormone somatotropo, mentre l'arginina inibisce il rilascio di somatostatina. Si tratta di un test molto potente. L'arginina viene infusa come esposto sopra. Al termine dell'infusione si somministra un bolo di GHRH alla dose di 1 \unit{\micro g}/kg (fino ad un massimo di 50 \unit{\micro g}) per via endovenosa. I prelievi ematici per dosare l'ormone somatotropo vengono eseguiti a 0-30-90 minuti. Gli effetti collaterali sono scarsi; può comparire un transitorio rossore al volto.

\subsubsection*{La screzione somatotropa spontanea media}
La secrezione spontanea media viene generalmente valutata nelle 12 ore notturne con prelievi ematici ogni trenta minuti. Nei bambini della prima infanzia ed in quelli di età maggiore le cui condizioni cliniche non sono soddisfacenti, i prelievi possono essere eseguiti ad intervalli di un'ora.

\subsubsection*{La generazione somatomedinica}
Il test di generazione somatomedinica si effettua con la somministrazione di rGH sottocute alla dose di 0,1 U.I./kg per quattro sere consecutive alle ore 21. La risposta al test è normale quando il valore di IGF-1 dopo rGH esogeno aumenta almeno del 50 \%
rispetto al valore basale.

\subsubsection*{}
Tutti i pazienti in esame sono stati sottoposti a valutazione ormonale. 

In 12 di questi sono stati riscontrati bassi livelli di GH in risposta a due test provocativi classici. I valori medi di GH erano compresi fra 5 e 10 \unit{\micro g}/l. Un solo paziente presentava deficit severo (entrambe le risposte erano inferiori a 5 \unit{\micro g}/l) ed alla RMN l'ipofisi è risultata ipoplasica.

2 pazienti hanno presentato insufficiente secrezione spontanea notturna di ormone somatotropo.

6 soggetti presentavano bassi livelli di somatomedine normoresponsivi al test di generazione.


In 2 pazienti sono stati riscontrati bassi livelli di GH in risposta al test di stimolo massimale.

Nessun paziente ha presentato deficit di ormone somatotropo severo persistente a patogenesi ipofisaria. 
