\chapter{Casistica, materiali e metodi}

\section{Casistica}

La mia casistica è composta da 28 pazienti (13 maschi e 15 femmine) nati SGA, affetti da bassa statura e trattati con rGH presso la Struttura Semplice di Auxologia del Dipartimento di Discipline Pediatriche e dell'Adolescenza dell'Università  di Torino. Tutti i soggetti in esame sono di provenienza italiana, nati a partire dall'anno 1983.

I criteri di inclusione nella casistica sono stati:

\begin{itemize}
\item peso e/o lunghezza alla nascita inferiori al terzo centile secondo gli standard antropometrici neonatali prodotti dalla task-force della Società Italiana di Neonatologia e basati su una popolazione dell'Italia settentrionale\cite{standard} (Figura~\ref{fig:StandardNeonataliNordOccidentali}); di tutti i pazienti sono note e attendibili l'età gestazionale e le principali misure neonatali
\begin{figure}[!h]
  \begin{center}
      \includegraphics{grafici/centili/centili} %\\
  \end{center}
  \caption{Standard antropometrici neonatali relativi a peso ed altezza nell'Italia nord-orientale}
  \label{fig:StandardNeonataliNordOccidentali}
\end{figure}
\item assenza di malformazioni maggiori, sindromi genetiche conosciute (ad esempio sindrome di Turner), sindrome di Silver Russell, malattie croniche che limitano la crescita (ad esempio insufficienza renale cronica, malattia celiaca), displasie scheletriche di rilievo. Infatti tutti i pazienti sono stati sottoposti ad esami ematochimici (emocromo, glicemia, creatininemia, dosaggio delle transaminasi epatiche plasmatiche, screening sierologico per celiachia, concentrazione ematica di proteina C reattiva, EGA venosa) al fine di escludere bassa statura da altre cause e si è valutato il cariotipo dei soggetti di sesso femminile
 \item anamnesi negativa per l'assunzione di farmaci che interferiscono con l'accrescimento (ad esempio corticosteroidi assunti ad alte dosi per un lungo periodo di tempo)
 \item assenza di traumi cranici
 \item assenza di anoressia nervosa
 \item anamnesi negativa per malnutrizione e disturbi psicosociali
 \item misurazione delle stature di entrambi i genitori con statimetro di Harpenden, ad eccezione dei genitori naturali di due soggetti adottati
\item trattamento con rGH per almeno 1,3 anni consecutivi
\item raggiungimento della statura definitiva, misurata con statimetro di Harpenden. 
\end{itemize}

La presenza o meno del deficit di GH non è stata considerata un criterio di inclusione dal momento che secondo numerosi studi la risposta alla terapia non è influenzata dallo status secretorio\cite{sas1999growth} \cite{van2003adult}.

Date le disposizioni vigenti allora e trattandosi di portatori di grave deficit staturale, tutti i pazienti della casistica sono stati sottoposti al dosaggio dell'ormone somatotropo. 

In otto di questi sono stati riscontrati insufficienti livelli di GH in risposta a due test provocativi classici (primo test con arginina, secondo test con clonidina o glucagone). I valori medi di GH erano compresi fra 5 e 10 \unit{\micro g}/l, trattandosi quindi di una lieve compromissione della produzione ormonale (il valore normale è superiore a 10\unit{\micro g}/l). Un solo paziente tra questi otto presentava deficit severo (entrambe le risposte erano inferiori a 5 \unit{\micro g}/l) ed alla RMN l'ipofisi è risultata ipoplasica. Sottoposto a fine crescita alla rivalutazione della funzione ipofisaria con test provocativo massimale (arginina + GHRH), il paziente ha presentato valori nella norma.

 Una dei ventotto pazienti ha presentato normale risposta ai test provocativi ed insufficiente secrezione spontanea notturna di ormone somatotropo, ma ai fini dell'analisi dei risultati è stata considerata non affetta da deficit di GH.
 
 Alcuni pazienti presentavano bassi livelli di somatomedine normoresponsivi al test di generazione ed hanno potuto accedere alla terapia secondo quanto previsto dalla nota AIFA 39 vigente allora.
 
 I rimanenti sono stati autorizzati al trattamento dalla Commissione Regionale preposta alla sorveglianza epidemiologica ed al monitoraggio dell'appropriatezza del trattamento con GH. 

\section{Materiali e metodi}

\subsection{Valutazione auxologica}
Tutti i pazienti sono stati sottoposti a valutazione auxologica prima dell'inizio della terapia, durante e al termine di questa . Infine sono stati valutati a fine crescita. Le misurazioni durante la terapia sono state effettuate a distanza di sei mesi una dall'altra.

\subsubsection*{Età decimale}
L'età è stata espressa in anni e decimi di anno, così da calcolare con precisione la velocità di crescita in cm/anno per intervalli di tempo variabili. Secondo il calendario decimale l'anno viene diviso in millesimi, così che ogni data corrisponde ad un numero decimale. Ad esempio il 1 maggio 1999 corrisponde a 99,329. La data di nascita viene registrata in modo analogo. L'età ad ogni misurazione corrisponderà alla differenza fra il decimale della data della misurazione e il decimale della data di nascita\cite{tanner1990foetus}.    


\subsubsection*{Statura eretta}
 La misurazione è stata effettuata con lo statimetro di Harpenden, secondo la tecnica di Cameron\cite{cameron1984measurement}
; la misura, espressa in centimetri e millimetri, è la media di tre misurazioni eseguite dallo stesso operatore. Le misurazioni così ottenute sono state confrontate con le carte di Tanner\cite{tanner1966standards}.
Inoltre la statura è stata espressa in SDS.

  
\subsubsection*{Peso}
 \'E stato rilevato mediante una bilancia tarata che assicurava un'approssimazione di kg 0,1. Tutti i soggetti sono stati pesati con il minimo di indumenti. Le misurazioni sono state confrontate con gli standard di Tanner\cite{tanner1966standards}. 

\subsubsection*{Velocità di crescita staturale}
Per calcolarla occorre dividere la differenza fra due misurazioni per l'intervallo di tempo fra queste misurazioni. L'intervallo non deve essere nè troppo breve nè molto lungo: i pazienti in terapia sono stati misurati ogni sei mesi. Sono stati utilizzati gli standard per la velocità di crescita staturale di Tanner\cite{tanner1966standards}.

\subsubsection*{Inizio pubertà}
Nella femmina l'inizio della pubertà è stato stabilito secondo la classificazione di Tanner\cite{tanner1990foetus} quando lo sviluppo mammario era allo stadio B2.

Nel maschio l'inizio della pubertà è stato definito con l'orchidometro di Prader quando veniva raggiunto il volume testicolare di 4 ml.


\subsubsection*{Statura bersaglio}
\'E  la statura più probabile per i figli sani di una coppia e corrisponde alla media aritmetica delle stature dei genitori corrette per il sesso. Il centro del bersaglio (Mid-Parental Height, MPH ) si ottiene nel modo seguente: per i maschi si sommano la statura del padre e quella della madre espressa al maschile (cioè si aggiungono 13 centimetri alla statura materna, perché tale è la differenza statistica fra maschi e femmine) e si divide il risultato per due; per le femmine si sommano la statura della madre e quella del padre espressa al femminile (vale a dire si sottraggono 13 centimetri alla statura paterna per il motivo sopra esposto) e si divide il risultato per due. 
La fascia--bersaglio (centro del bersaglio $\pm$8 cm, cioè $\pm$2 DS) fornisce l'intervallo entro cui dovrebbe situarsi la maggior parte dei figli di una coppia \cite{tanner1990foetus}.
La statura bersaglio è stata espressa in centimetri e in SDS.


\subsubsection*{Maturazione scheletrica}
Fornisce, insieme agli stadi puberali e all'orchidometria, importanti informazioni sull'età biologica del paziente. L'età ossea viene assegnata secondo il sistema TW2 di Tanner e Whitehouse\cite{tanner1983assessment} sulla radiografia della mano e del polso di sinistra. La lettura è stata eseguita fin dall'inizio dello studio dagli stessi due operatori.


\subsubsection*{Statura definitiva}
La statura raggiunta veniva ritenuta definitiva (Final Height, FH) quando la velocità di crescita risultava inferiore ai 2 cm/anno o l'età ossea del paziente era di almeno 15 anni nella femmina, 16 nel maschio.
Ai fini della presentazione grafica e dell'elaborazione statistica la statura definitiva è stata espressa in SDS.




\subsection{Analisi Statistica}
Si è utilizzato un modello di regressione lineare multivariata per valutare l'effetto di diversi parametri predittori (età a inizio terapia; statura a inizio terapia; statura a inizio pubertà; durata prepuberale della terapia; durata totale della terapia; deficit di GH) sul guadagno staturale considerato come variabile dipendente. 
Questo modello, costruito con i predittori sopraelencati, rende conto del 77,1\% del guadagno staturale ($r^2 = 0.771$).
La tabella 4.7 nel capitolo Risultati mostra l'esito dell'analisi multivariata. 
